\subsection{Destructuring Let with Type Annotated Composite Patterns}
\label{sec:calculus-let}

To ease the use of products and other datatypes, many languages feature destructuring bindings.
In a typed setting, we may want to granularly add type annotations in patterns as well.
In a bidirectional setting, as it turns out, this combination of features is surprisingly tricky to get right!
Let us add let expressions and simple patterns, writing $\PWild$ for the wildcard pattern, $\PVar{x}$ for a variable pattern, and $\PPair{\PMV_1}{\PMV_2}$ for a pair:
%
\[\begin{array}{rrcl}
  \EMName  & \EMV  & \Coloneqq & \cdots \mid \ELet{\PMV}{\EMV}{\EMV} \\
  \PMName  & \PMV  & \Coloneqq & \PWild \mid \PVar{x} \mid \PPair{\PMV}{\PMV} \mid \PAsc{\PMV}{\TMV}
\end{array}\]
%

Consider the following program:
\[%
  \ELet{\PPair{\PVar{a}}{\PVar{b}}}{\EPair{1}{2}}{\EMV}
\]%

To type this, one approach is to synthesize a type from the pattern and analyze the
definition against that type. However, this may run afoul of user expectation, which might
reasonably suppose it to be equivalent to the expanded expression
$\ELet{\PVar{a}}{1}{\ELet{\PVar{b}}{2}{\EMV}}$. In the original, the pattern synthesizes the type
$\TProd{\TUnknown}{\TUnknown}$, and $\EPair{1}{2}$ is analyzed against it, hence $1$ and $2$ are
each analyzed against $\TUnknown$. In the expanded version, however, they are typed in synthetic
mode.

Though it is benign in this example, there is a subtle semantic distinction: synthetic
mode imposes \emph{no} type constraints, whereas analytic mode imposes a \emph{trivial} type
constraint. This manifests when expressions may have internal type inconsistencies,  as in the
following:
\[%
  \ELet{\PVar{a}}{\EIf{\ETrue}{1}{\EFalse}}{\EMV}
\]%
If the pattern $\PVar{a}$ suggests that the conditional is in synthetic position, it will be marked
with an inconsistent branches error mark following our development above. If instead it is in
analytic position against the unknown type, each branch is checked independently against this
trivial constraint, and no mark will be produced.

To remedy this situation and preserve the semantic distinction between synthesis and analysis
against the unknown type, we introduce a pattern annotation form, written $\PAsc{\PMV}{\TMV}$, which
allows the explicit imposition of typing constraints. That is, the absence of any type annotation on
a variable pattern places the corresponding definition in synthetic mode, while the programmer may
impose typing constraints---the trivial constraint, if they wish---on the definition.
As illustration, consider the following program:
\[%
  \ELet{\PPair{\PVar{a}}{\PAsc{\PVar{b}}{\TUnknown}}}{\EPair{\EIf{\ETrue}{1}{\EFalse}}{\EIf{\ETrue}{2}{\EFalse}}}{\EMV}
\]%
Since $\PVar{a}$ has no constraint, the left component is in synthetic position, leading to an
inconsistent branches error. The type annotation on $\PVar{b}$ puts the right component in analytic
mode against the unknown type, leading to no error marks.

This is achieved by adding a ``switch type'', denoted $\TUnknownSwitch$, which exists entirely to trigger a ``switch'' to synthesis.
$\TUnknownSwitch$ represents a trivial constraint, instantiated as a type for the sole purpose of using the existing machinery to propagate the appropriate mode switch to the relevant sub-expression of the definition.
This is expressed by the following rule:
%
\begin{mathpar}
  \judgment{
    \ctxSynTypeU{\ctx}{\EMV}{\TMV}
  }{
    \ctxAnaTypeU{\ctx}{\EMV}{\TUnknownSwitch}
  }{UASynSwitch}
\end{mathpar}
%
This switch type carries over identically to a non-gradual setting, but here we represent it as a variant of $\TUnknown$ as it behaves identically with respect to consistency and our notions of matched arrow/product. 
% Note that in a non-gradual setting, the matched product rule (or equivalent) would still be required to support the cases such as $\ELet{\PVar{a}}{\EPair{1}{2}}{a}$.

In the previous example, the unannotated pattern variable $\PVar{a}$ synthesizes
$\TUnknownSwitch$, and the annotated pattern $\PAsc{\PVar{b}}{\TUnknown}$ synthesizes $\TUnknown$.
We write $\ctxSynPatU{\ctx}{\PMV}{\TMV}$ to say that the pattern $\PMV$ synthesizes type $\TMV$.
%
\begin{mathpar}
  \judgment{ }{
    \ctxSynPatU{\ctx}{\PVar{x}}{\TUnknownSwitch}
  }{USPVar}

  \inferrule[USPPair]{
    \ctxSynPatU{\ctx}{\PMV_1}{\TMV_1} \\
    \ctxSynPatU{\ctx}{\PMV_2}{\TMV_2}
  }{
    \ctxSynPatU{\ctx}{\PPair{\PMV_1}{\PMV_2}}{\TProd{\TMV_1}{\TMV_2}}
  }
 
  \judgment{
    \ctxAnaPatU{\ctx}{\PMV}{\TMV}{\ctx'}
  }{
    \ctxSynPatU{\ctx}{\PAsc{\PMV}{\TMV}}{\TMV}
  }{USPAnn}
\end{mathpar}
 
Given by the judgment $\ctxAnaPatU{\ctx}{\PMV}{\TMV}{\ctx'}$, patterns are also typed analytically,
in which case they produce an output context $\ctx'$ that extends $\ctx$ with bindings introduced
by the pattern. Note again that $\TUnknownSwitch$ exists entirely to ensure that sub-expressions of the
definition are assigned an appropriate typing mode---it is never added to the context, i.e., it
cannot escape into the rest of the program and cause body expressions to synthesize accidentally.
%
\begin{mathpar}
  \judgment{ }{
    \ctxAnaPatU{\ctx}{\PVar{x}}{\TMV}{\extendCtx{\ctx}{x}{\TMV}}
  }{UAPVar}

  \inferrule[UAPPair]{
    \matchedProd{\TMV}{\TMV_1}{\TMV_2} \\\\
    \ctxAnaPatU{\ctx}{\PMV_1}{\TMV_1}{\ctx_1} \\
    \ctxAnaPatU{\ctx_1}{\PMV_2}{\TMV_2}{\ctx_2}
  }{
    \ctxAnaPatU{\ctx}{\PPair{\PMV_1}{\PMV_2}}{\TMV}{\ctx_2}
  }
 
  \judgment{
    \ctxAnaPatU{\ctx}{\PMV}{\TMV'}{\ctx'} \\\\
    \consistent{\TMV}{\TMV'}
  }{
    \ctxAnaPatU{\ctx}{\PAsc{\PMV}{\TMV'}}{\TMV}{\ctx'}
  }{UAPAnn}
\end{mathpar}

Synthesis for let expressions is straightforward: we synthesize the type of the definition and analyze the pattern against it. We then synthesize the body under the new context:
\[%
  \judgment{
    %\ctxSynPatU{\ctx}{\PMV}{\TMV_p} \\
    %\ctxAnaTypeU{\ctx}{\EMV_{1}}{\TMV_p} \\
    \ctxSynTypeU{\ctx}{\EMV_{1}}{\TMV_{1}} \\
    \ctxAnaPatU{\ctx}{\PMV}{\TMV_{1}}{\ctx'} \\
    \ctxSynTypeU{\ctx'}{\EMV_{2}}{\TMV_2}
  }{
    \ctxSynTypeU{\ctx}{\ELet{\PMV}{\EMV_{1}}{\EMV_{2}}}{\TMV_2}
  }{USLetPat}
\]%
The analytic rule is similar, with the final premise and conclusion changed to analysis.

Unfortunately, attempting an analogous marking rule runs into some trouble.
We want any type annotations in the pattern to be canonical, which means analyzing the definition to attribute to it any inconsistencies.
But we still must analyze the pattern, incorporating the definition's type to produce the context needed by the body.
So we do a ``round trip'': analyze the definition against the pattern's type, and then the pattern against the definition's.
Since the first analysis establishes consistency, this second analysis is guaranteed to succeed; we only care about the context produced:
\[%
  \judgment{
    \ctxSynFixedInto{\ctx}{\PMV}{\PCMV}{\TMV_p} \\
    \ctxAnaFixedInto{\ctx}{\EMV_{1}}{\ECMV_{1}}{\TMV_{p}} \\
    \ctxSynTypeU{\ctx}{\EMV_{1}}{\TMV_{1}} \\\\
    \ctxAnaPatU{\ctx}{\PMV}{\TMV_{1}}{\ctx'} \\
    \ctxSynFixedInto{\ctx'}{\EMV_{2}}{\ECMV_{2}}{\TMV_2}
  }{
    \ctxSynFixedInto{\ctx}{\ELet{\PMV}{\EMV_{1}}{\EMV_{2}}}{\ELet{\PCMV}{\ECMV_{1}}{\ECMV_{2}}}{\TMV_2}
  }{MKSLetPat}
\]%

We omit the marking of patterns---they
may be derived in the same way as those for expressions and are governed by similar metatheorems (again, totality guides the derivation).
In the analytic cases, we introduce error marks that parallel to $\ECInconType{\ECMV}$ and 
$\ECAnaNonMatchedProd{\ECMV}$.

Note that both \textsc{USLetPat} and \textsc{MKSLetPat} assume that the definition synthesizes.
In our system this is always the case, but if we added, for example, unannotated lambda abstractions, we might wish to relax this restriction.
We can freely add variations of the rules where if the definition fails to synthesize, it is instead analyzed against the synthesized type of the pattern.
To avoid an additional redundant analysis, we would further want to modify the pattern synthesis judgement to produce a context as well.

Recent work by \citet{yuan2023} addresses the problem of general pattern matching with typed holes,
providing mechanisms to reason statically about redundancy and exhaustiveness in the presence of
pattern holes. A comprehensive account incorporating the marked lambda calculus would employ error
marks indicating inexhaustive matches and redundant patterns.
