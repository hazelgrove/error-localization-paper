\subsection{Destructuring Let with Type Annotated Composite Patterns}
\label{sec:calculus-let}

To ease the use of products, many languages feature destructuring bindings. In a typed setting, we may want to granularly add type annotations in patterns as well. As it turns out this combination of features is surprisingly tricky to get right! Let us add let
expressions and simple patterns, writing $\PWild$ for the wildcard pattern, $\PVar{x}$ for a
variable pattern, and $\PPair{\PMV_1}{\PMV_2}$ for a pair. 
%
\[\begin{array}{rrcl}
  \EMName  & \EMV  & \Coloneqq & \cdots \mid \ELet{\PMV}{\EMV}{\EMV} \\
  \PMName  & \PMV  & \Coloneqq & \PWild \mid \PVar{x} \mid \PPair{\PMV}{\PMV} \mid \PAsc{\PMV}{\TMV}
\end{array}\]
%

Consider the following program:
\[%
  \ELet{\PPair{\PVar{a}}{\PVar{b}}}{\EPair{1}{2}}{\EMV}
\]%

To type this, the most obvious approach is to synthesize a type from the pattern and analyze the
definition against that type. However, this may run afoul of user expectation, which might
reasonably suppose it to be equivalent to the expanded expression
$\ELet{\PVar{a}}{1}{\ELet{\PVar{b}}{2}{\EMV}}$. In the original, the pattern synthesizes the type
$\TProd{\TUnknown}{\TUnknown}$, and $\EPair{1}{2}$ is analyzed against it, hence $1$ and $2$ are
each analyzed against $\TUnknown$. In the expanded version, however, they are typed in synthetic
mode.

Though it is benign in this example, there is a subtle semantic distinction: synthetic
mode imposes \emph{no} type constraints, whereas analytic mode imposes a \emph{trivial} type
constraint. This manifests when expressions may have internal type inconsistencies,  as in the
following:
\[%
  \ELet{\PVar{a}}{\EIf{\ETrue}{1}{\EFalse}}{\EMV}
\]%
If the pattern $\PVar{a}$ suggests that the conditional is in synthetic position, it will be marked
with an inconsistent branches error following our development of marking above. If instead it is in
analytic position against the unknown type, no mark will be produced.

To remedy this situation and preserve the semantic distinction between synthesis and analysis
against the unknown type, we introduce a pattern annotation form, written $\PAsc{\PMV}{\TMV}$, which
allows the explicit imposition of typing constraints. That is, the absence of any type annotation on
a variable pattern places the corresponding definition in synthetic mode, while the programmer may
impose typing constraints---the trivial constraint, if they wish---on the definition.

As illustration, consider the following program:
\[%
  \ELet{\PPair{\PVar{a}}{\PAsc{\PVar{b}}{\TUnknown}}}{\EPair{\EIf{\ETrue}{1}{\EFalse}}{\EIf{\ETrue}{2}{\EFalse}}}{\EMV}
\]%
Since $\PVar{a}$ has no constraint, the left component is in synthetic position, leading to an
inconsistent branches error. The type annotation on $\PVar{b}$ puts the right component in analytic
mode against the unknown type, leading to no error marks.

This system is achieved by augmenting the type system with a variant of the unknown type, written
$\TUnknownSwitch$, that triggers a ``switch'' to synthesis and otherwise behaves identically to
$\TUnknown$. In the previous example, the unannotated pattern variable $\PVar{a}$ synthesizes
$\TUnknownSwitch$, and the annotated pattern $\PAsc{\PVar{b}}{\TUnknown}$ synthesizes $\TUnknown$.
We write $\ctxSynPatU{\ctx}{\PMV}{\TMV}$ to say that the pattern $\PMV$ synthesizes type $\TMV$.
%
\begin{mathpar}
  \judgment{
    \ctxSynTypeU{\ctx}{\EMV}{\ctx'}
  }{
    \ctxAnaTypeU{\ctx}{\EMV}{\TUnknownSwitch}
  }{UASynSwitch}

  \judgment{ }{
    \ctxSynPatU{\ctx}{\PVar{x}}{\TUnknownSwitch}
  }{USPVar}
 
  \judgment{
    \ctxAnaPatU{\ctx}{\PMV}{\TMV}{\ctx'}
  }{
    \ctxSynPatU{\ctx}{\PAsc{\PMV}{\TMV}}{\TMV}
  }{USPAnn}
\end{mathpar}
 
Given by the judgment $\ctxAnaPatU{\ctx}{\PMV}{\TMV}{\ctx'}$, patterns are also typed analytically,
in which case they produce an output context $\ctx'$, which extends $\ctx$ with bindings introduced
by the pattern. Note that $\TUnknownSwitch$ exists entirely to ensure that sub-expressions of the
definition are assigned an appropriate typing mode---it is never added to the context, i.e. it
cannot escape into the rest of the program and cause body expressions to synthesize accidentally.
%
\begin{mathpar}
  \judgment{ }{
    \ctxAnaPatU{\ctx}{\PVar{x}}{\TMV}{\extendCtx{\ctx}{x}{\TMV}}
  }{UAPVar}
 
  \judgment{
    \ctxAnaPatU{\ctx}{\PMV}{\TMV'}{\ctx'} \\
    \consistent{\TMV}{\TMV'}
  }{
    \ctxAnaPatU{\ctx}{\PAsc{\PMV}{\TMV'}}{\TMV}{\ctx'}
  }{UAPAnn}
\end{mathpar}

To ensure propagation of type information between pattern, definition, and body, the destructuring
let requires some extra machinery.
\[%
  \judgment{
    \ctxSynPatU{\ctx}{\PMV}{\TMV_p} \\
    \ctxAnaTypeU{\ctx}{\EMV_{1}}{\TMV_p} \\
    \ctxSynTypeU{\ctx}{\EMV_{1}}{\TMV_{1}} \\\\
    \ctxAnaPatU{\ctx}{\PMV}{\TMV_{1}}{\ctx'} \\
    \ctxSynTypeU{\ctx'}{\EMV_{2}}{\TMV_2}
  }{
    \ctxSynTypeU{\ctx}{\ELet{\PMV}{\EMV_{1}}{\EMV_{2}}}{\TMV_2}
  }{USLetPat}
\]%
First, to ensure that the pattern and definition types are consistent, the pattern synthesizes a
type, against which the definition is analyzed. Then, the definition synthesizes a type, against
which the pattern is analyzed. This analysis is guaranteed to succeed; we need the context it
produces for use in the synthesis of the body's type. The analytic rule is similar, with the final
premise and conclusion changed to analysis.

The marking rule is directly analogous and utilizes the marking of patterns, which we omit---they
may be derived in the same way as those for expressions. In the analytic cases, we introduce error
marks that parallel to $\ECInconType{\ECMV}$ and $\ECAnaNonMatchedProd{\ECMV}$.
\[%
  \judgment{
    \ctxSynFixedInto{\ctx}{\PMV}{\PCMV}{\TMV_p} \\
    \ctxAnaFixedInto{\ctx}{\EMV_{1}}{\ECMV_{1}}{\TMV_{p}} \\
    \ctxSynTypeU{\ctx}{\EMV_{1}}{\TMV_{1}} \\\\
    \ctxAnaPatU{\ctx}{\PMV}{\TMV_{1}}{\ctx'} \\
    \ctxSynFixedInto{\ctx'}{\EMV_{2}}{\ECMV_{2}}{\TMV_2}
  }{
    \ctxSynFixedInto{\ctx}{\ELet{\PMV}{\EMV_{1}}{\EMV_{2}}}{\ELet{\PCMV}{\ECMV_{1}}{\ECMV_{2}}}{\TMV_2}
  }{ISLetPat}
\]%

% TODO conclude briefly?
