\newpage
\section{Appendix C: An extended syntax of PossibleTypeSets}

\begin{figure}[h!]
\centering
\hrule
$\arraycolsep=4pt\begin{array}{lll}
BinaryOperator~~ \square & ::=~~~~~
\rightarrow  ~\vert~ 
\times
\\
PotentialTypeSet~~ s & ::=
single(t) ~\vert~
cons(t, s)
\\
PotentialType~~ t & ::= 
  \tnum ~\vert~
  \tbool ~\vert~
  \TVar ~\vert~
  \tehole^p ~\vert~
  s ~\square~ s
  \\
\end{array}$
\label{fig:syntax_possible_type_sets}
\caption{Extended syntax of PotentialTypeSets and PotentialTypes}
\vspace{5px}
\hrule
\[\begin{array}{rcl}
    single(s_1 ~\square_{\alpha}~ s_2) ~\amalg~ single(s_3 ~\square_{\alpha}~ s_4) & = & single((s_1 ~\amalg~ s_3) ~\square_{\alpha}~ (s_2 ~\amalg~ s_4)) \\
    single(t) ~\amalg~ single(t') & = & cons(t, single(t')) \\
    cons(t,s) ~\amalg~ single(t) & = & cons(t, s) \\
    cons(s_1 ~\square_{\alpha}~ s_2, s) ~\amalg~ single(s_3 ~\square_{\alpha}~ s_4) & = & cons((s_1 ~\amalg~ s_3) ~\square_{\alpha}~ (s_2 ~\amalg~ s_4), s) \\
    cons(t,s) ~\amalg~ cons(t',s') & = & cons(t,s) ~\amalg~ single(t') ~\amalg~ s' \\
\end{array}\] 
\caption{Extended rules for merging PotentialTypeSets}
\vspace{5px} 
\hrule
\label{fig:extended_merging_possible_type_sets}
\vspace{-5px}
\end{figure}

Here, we take a more general approach to representing binary types. Namely, a \emph{PotentialType} may be the combination of two \emph{PotentialType}s through any \emph{BinaryOperator} $\square$. The rules for extension follow the same logic as before, where we enforce that extension of a \emph{PotentialTypeSet} containing $s_1 ~\square_{\alpha}~ s_2$ with $s_3 ~\square_{\alpha}~ s_4$ simply merges the left and right \emph{PotentialTypeSet}s, thus yielding $(s_1 ~\amalg~ s_3) ~\square_{\alpha}~ (s_2 ~\amalg~ s_4)$. Additionally, we choose to represent type variables as a ground type, thus enforcing that type variables are inconsistent with other ground types and consistent with type holes and each other. 