%
% utilities
%
\newcommand{\colorSideJudge}{Black!50}
\newcommand{\colorCText}{\colorUText}
\newcommand{\colorCBkgSyn}{\colorUBkgSyn}
\newcommand{\colorCBkgAna}{\colorUBkgAna}
\definecolor{hole}{RGB}{162,85,162}
\definecolor{cursorhighlight}{RGB}{230,255,230}
\definecolor{cursor}{RGB}{76,170,76}
\newcommand{\mathcolorbox}[2]{{\fboxsep=1pt\colorbox{#1}{$\displaystyle #2$}}}

%
% color
%
\newtcolorbox{highlightbox}[2][]{
  on line,
  hbox,
  boxsep=0pt,
  left=1pt,
  right=1pt,
  top=1pt,
  bottom=1pt,
  colframe=white,
  colback=#2
  #1,
}

\DeclareMathOperator{\?}{?}
\DeclareMathOperator{\dom}{dom}


\newcommand{\colorOkSideJudge}{Black!50}
\newcommand{\colorFailSideJudge}{red}

%
% relations
%

% equality
\newcommand{\equal}[2]{\ensuremath{#1 = #2}}
\newcommand{\notEqual}[2]{\ensuremath{#1 \neq #2}}

% consistency
\newcommand{\consistentRel}{\ensuremath{\sim}}
\newcommand{\consistent}[2]{\ensuremath{#1 \goodcolor{\colorOkSideJudge}{\consistentRel} #2}}
\newcommand{\inconsistentRel}{\ensuremath{\nsim}}
\newcommand{\inconsistent}[2]{\ensuremath{\goodcolor{\colorFailSideJudge}{#1 \inconsistentRel #2}}}

% matching
\newcommand{\matchedRel}[1]{\ensuremath{\blacktriangleright_{#1}}}
\newcommand{\notMatchedRel}[1]{\ensuremath{\blacktriangleright_{\centernot#1}}}
\newcommand{\matchedArrowRel}{\ensuremath{\matchedRel{\to}}}
\newcommand{\notMatchedArrowRel}{\ensuremath{\notMatchedRel{\to}}}
\newcommand{\matchedArrow}[3]{\ensuremath{#1 \goodcolor{\colorOkSideJudge}{\matchedArrowRel} \TArrow{#2}{#3}}}
\newcommand{\notMatchedArrow}[1]{\ensuremath{\goodcolor{\colorFailSideJudge}{#1 \notMatchedArrowRel}}}
\newcommand{\matchedProdRel}{\ensuremath{\matchedRel{\times}}}
\newcommand{\notMatchedProdRel}{\ensuremath{\notMatchedRel{\times}}}
\newcommand{\matchedProd}[3]{\ensuremath{#1 \goodcolor{\colorOkSideJudge}{\matchedProdRel} \TProd{#2}{#3}}}
\newcommand{\notMatchedProd}[1]{\ensuremath{\goodcolor{\colorFailSideJudge}{#1 \notMatchedProdRel}}}

% base types
\newcommand{\base}[1]{\ensuremath{#1 ~{\normalfont\textsf{base}}}}

% 
% syntax
%
\newcommand{\TMName}{{\normalfont\textsf{Type}}}
\newcommand{\TMSet}{\ensuremath{T}}
\newcommand{\TMV}{\ensuremath{\tau}}

\newcommand{\joinRel}{\sqcup}
\newcommand{\noJoinRel}{\centernot\sqcup}
\newcommand{\TJoin}[2]{\ensuremath{#1 \joinRel #2}}

\newcommand{\TUnknown}{\ensuremath{\?}}
\newcommand{\TUnknownSwitch}{\ensuremath{\?^{\Rightarrow}}}
\newcommand{\TNum}{\ensuremath{{\normalfont\textsf{num}}}}
\newcommand{\TBool}{\ensuremath{{\normalfont\textsf{bool}}}}
\newcommand{\TArrow}[2]{\ensuremath{#1 \to #2}}
\newcommand{\TProd}[2]{\ensuremath{#1 \times #2}}

%
% contexts
%
\newcommand{\ctx}{\ensuremath{\Gamma}}
\newcommand{\extendCtx}[3]{\ensuremath{#1 , ~\assignType{#2}{#3}}}
\newcommand{\inCtx}[3]{\ensuremath{\assignType{#2}{#3} \in #1}}
\newcommand{\notInCtx}[2]{\ensuremath{\goodcolor{\colorFailSideJudge}{#2 \not\in \dom(#1)}}}
\newcommand{\withCtx}[2]{\ensuremath{#1 \vdash #2}}

%
% typing
%
\newcommand{\assignType}[2]{\ensuremath{#1 : #2}}
\newcommand{\synType}[2]{\ensuremath{#1 \Rightarrow #2}}
\newcommand{\notSynType}[2]{\ensuremath{#1 \not\Rightarrow #2}}
\newcommand{\ctxSynType}[3]{\ensuremath{\withCtx{#1}{\synType{#2}{#3}}}}
\newcommand{\ctxNotSynType}[3]{\ensuremath{\withCtx{#1}{\notSynType{#2}{#3}}}}
\newcommand{\anaType}[2]{\ensuremath{#1 \Leftarrow #2}}
\newcommand{\notAnaType}[2]{\ensuremath{#1 \goodcolor{\colorFailSideJudge}{\not\Leftarrow} #2}}
\newcommand{\ctxAnaType}[3]{\ensuremath{\withCtx{#1}{\anaType{#2}{#3}}}}
\newcommand{\ctxNotAnaType}[3]{\ensuremath{\withCtx{#1}{\notAnaType{#2}{#3}}}}

% !requires types

%
% external expressions
%
\newcommand{\EMName}{{\normalfont\textsf{UExp}}}
\newcommand{\EMSet}{\ensuremath{E}}
\newcommand{\EMV}{\ensuremath{e}}

% holes
\newcommand{\EEHole}{\ensuremath{\ECEHole}}

% integers
\newcommand{\ENum}[1]{\ensuremath{\ECNum{#1}}}
\newcommand{\ENumMV}{\ensuremath{\ECNumMV}}
\newcommand{\EOpPlus}{\ensuremath{\ECOpPlus}}
\newcommand{\EPlus}[2]{\ensuremath{\ECPlus{#1}{#2}}}

% booleans
\newcommand{\ETrue}{\ensuremath{\ECTrue}}
\newcommand{\EFalse}{\ensuremath{\ECFalse}}
\newcommand{\EIf}[3]{\ensuremath{\ECIf{#1}{#2}{#3}}}

% lambdas
\newcommand{\ELam}[3]{\ensuremath{\ECLam{#1}{#2}{#3}}}
\newcommand{\EAp}[2]{\ensuremath{\ECAp{#1}{#2}}}

% pairs
\newcommand{\EPair}[2]{\ensuremath{\ECPair{#1}{#2}}}
\newcommand{\EProjL}[1]{\ensuremath{\ECProjL{#1}}}
\newcommand{\EProjR}[1]{\ensuremath{\ECProjR{#1}}}

% let
\newcommand{\ELet}[3]{\ensuremath{\ECLet{#1}{#2}{#3}}}

%
% marked expressions
%
\newcommand{\ECMName}{{\normalfont\textsf{MExp}}}
\newcommand{\ECMSet}{\ensuremath{\check{E}}}
\newcommand{\ECMV}{\ensuremath{\check{e}}}

% holes
\definecolor{hole}{RGB}{162,85,162}
\newcommand{\ECEHole}{\ensuremath{\textcolor{hole}{\bm{\llparenthesis}}\textcolor{hole}{\bm{\rrparenthesis}}}}

% integers
\newcommand{\ECNum}[1]{\ensuremath{\underline{#1}}}
\newcommand{\ECNumMV}{\ensuremath{\ECNum{n}}}
\newcommand{\ECOpPlus}{\ensuremath{+}}
\newcommand{\ECPlus}[2]{\ensuremath{#1 \ECOpPlus #2}}

% booleans
\newcommand{\ECTrue}{\ensuremath{{\normalfont\textsf{tt}}}}
\newcommand{\ECFalse}{\ensuremath{{\normalfont\textsf{ff}}}}
\newcommand{\ECIf}[3]{\ensuremath{\textsf{if}~ #1 ~\textsf{then}~ #2 ~\textsf{else}~ #3}}

% lambdas
\newcommand{\ECLam}[3]{\ensuremath{\lambda #1 : #2. ~#3}}
\newcommand{\ECAp}[2]{\ensuremath{#1 ~#2}}

% pairs
\newcommand{\ECPair}[2]{\ensuremath{(#1, #2)}}
\newcommand{\ECProjL}[1]{\ensuremath{\pi_1 #1}}
\newcommand{\ECProjR}[1]{\ensuremath{\pi_2 #1}}

% let
\newcommand{\ECLet}[3]{\ensuremath{\textsf{let}~ #1 = #2 ~\textsf{in}~ #3}}

% errors
\newcommand{\MRFree}{\ensuremath{\mathsmaller{\mathsmaller{\mathsmaller{\square}}}}}
\newcommand{\MRInconType}{\ensuremath{\mathsmaller{\inconsistentRel}}}
\newcommand{\MRInconBr}{\ensuremath{\mathlarger{\noMeetRel}}}
\newcommand{\MRInconAsc}{\ensuremath{\bm{:}}}
\newcommand{\MRSynNonMatchedArrow}{\ensuremath{\mathsmaller{\notMatchedArrowRel}}}
\newcommand{\MRSynNonMatchedProd}{\ensuremath{\mathsmaller{\notMatchedProdRel}}}
\newcommand{\MRAnaNonMatchedArrow}{\ensuremath{\mathsmaller{\notMatchedArrowRel}}}
\newcommand{\MRAnaNonMatchedProd}{\ensuremath{\mathsmaller{\notMatchedProdRel}}}
\newcommand{\MRFreeTypeVar}{\ensuremath{\MRFree}}

\newcommand{\MRSyn}{\ensuremath{\mathsmaller{\Rightarrow}}}
\newcommand{\MRAna}{\ensuremath{\mathsmaller{\Leftarrow}}}

\newcommand{\ECMarked}[3]{\ensuremath{\textcolor{red}{\bm{\llparenthesis}#1\bm{\rrparenthesis}_{#2}^{#3}}}}
\newcommand{\ECMarkedFree}[1]{\ensuremath{\ECMarked{#1}{\MRFree}{}}}
\newcommand{\ECMarkedInconType}[1]{\ensuremath{\ECMarked{#1}{\MRInconType}{}}}
\newcommand{\ECMarkedInconBr}[1]{\ensuremath{\ECMarked{#1}{\MRInconBr}{}}}
\newcommand{\ECMarkedInconAsc}[1]{\ensuremath{\ECMarked{#1}{\MRInconAsc}{}}}
\newcommand{\ECMarkedSynNonMatchedArrow}[1]{\ensuremath{\ECMarked{#1}{\MRSynNonMatchedArrow}{\MRSyn}}}
\newcommand{\ECMarkedSynNonMatchedProd}[1]{\ensuremath{\ECMarked{#1}{\MRSynNonMatchedProd}{\MRSyn}}}
\newcommand{\ECMarkedAnaNonMatchedArrow}[1]{\ensuremath{\ECMarked{#1}{\MRAnaNonMatchedArrow}{\MRAna}}}
\newcommand{\ECMarkedAnaNonMatchedProd}[1]{\ensuremath{\ECMarked{#1}{\MRAnaNonMatchedProd}{\MRAna}}}

\newcommand{\ECFree}[1]{\ensuremath{\ECMarkedFree{#1}}}
\newcommand{\ECInconType}[1]{\ensuremath{\ECMarkedInconType{#1}}}
\newcommand{\ECInconBr}[3]{\ensuremath{\ECMarkedInconBr{\ECIf{#1}{#2}{#3}}}}
\newcommand{\ECInconAsc}[1]{\ensuremath{\ECMarkedInconAsc{#1}}}
\newcommand{\ECSynNonMatchedArrow}[1]{\ensuremath{\ECMarkedSynNonMatchedArrow{#1}}}
\newcommand{\ECSynNonMatchedProd}[1]{\ensuremath{\ECMarkedSynNonMatchedProd{#1}}}
\newcommand{\ECAnaNonMatchedArrow}[1]{\ensuremath{\ECMarkedAnaNonMatchedArrow{#1}}}
\newcommand{\ECAnaNonMatchedProd}[1]{\ensuremath{\ECMarkedAnaNonMatchedProd{#1}}}

\newcommand{\ECLamInconAsc}[3]{\ensuremath{\ECInconAsc{\ECLam{#1}{#2}{#3}}}}
\newcommand{\ECApSynNonMatchedArrow}[2]{\ensuremath{\ECAp{\ECSynNonMatchedArrow{#1}}{#2}}}
\newcommand{\ECProjLSynNonMatchedProd}[1]{\ensuremath{\ECProjL{\ECSynNonMatchedProd{#1}}}}
\newcommand{\ECProjRSynNonMatchedProd}[1]{\ensuremath{\ECProjR{\ECSynNonMatchedProd{#1}}}}
\newcommand{\ECLamAnaNonMatchedArrow}[3]{\ensuremath{\ECAnaNonMatchedArrow{\ECLam{#1}{#2}{#3}}}}
\newcommand{\ECPairAnaNonMatchedProd}[2]{\ensuremath{\ECAnaNonMatchedProd{\ECPair{#1}{#2}}}}

% !requires types

%
% typing
%

% unmarked
\newcommand{\colorUText}{PineGreen!90}
\newcommand{\colorUBkgSyn}{Gray!5}
\newcommand{\colorUBkgAna}{Gray!5}

\newcommand{\withCtxU}[2]{\ensuremath{#1 \goodcolor{\colorUText}{\vdash_{\hspace{-3.25pt}\scaleto{U}{3.5pt}}} #2}}
\newcommand{\synTypeU}[2]{\ensuremath{#1 \goodcolor{\colorUText}{\Rightarrow} #2}}
\newcommand{\notSynTypeU}[2]{\ensuremath{#1 \not\Rightarrow #2}}
\newcommand{\ctxSynTypeU}[3]{
  \begin{highlightbox}{\colorUBkgSyn}
    \ensuremath{\withCtxU{#1}{\synTypeU{#2}{#3}}}
  \end{highlightbox}}
\newcommand{\ctxNotSynTypeU}[3]{\ensuremath{\withCtxU{#1}{\notSynTypeU{#2}{#3}}}}

\newcommand{\anaTypeU}[2]{\ensuremath{#1 \goodcolor{\colorUText}{\Leftarrow} #2}}
\newcommand{\notAnaTypeU}[2]{\ensuremath{#1 \goodcolor{\colorFailSideJudge}{\not\Leftarrow} #2}}
\newcommand{\ctxAnaTypeU}[3]{
  \begin{highlightbox}{\colorUBkgAna}
    \ensuremath{\withCtxU{#1}{\anaTypeU{#2}{#3}}}
  \end{highlightbox}}
\newcommand{\ctxNotAnaTypeU}[3]{\ensuremath{\withCtxU{#1}{\notAnaTypeU{#2}{#3}}}}

% marked
\newcommand{\colorMText}{DarkOrchid}
\newcommand{\colorMBkgSyn}{Gray!5}
\newcommand{\colorMBkgAna}{Gray!5}

\newcommand{\withCtxM}[2]{\ensuremath{#1 \goodcolor{\colorMText}{\vdash_{\hspace{-3.75pt}\scaleto{M}{3.5pt}}} #2}}
\newcommand{\synTypeM}[2]{\ensuremath{#1 \goodcolor{\colorMText}{\Rightarrow} #2}}
\newcommand{\notSynTypeM}[2]{\ensuremath{#1 \not\Rightarrow #2}}
\newcommand{\ctxSynTypeM}[3]{
  \begin{highlightbox}{\colorMBkgSyn}
    \ensuremath{\withCtxM{#1}{\synTypeM{#2}{#3}}}
  \end{highlightbox}}
\newcommand{\ctxNotSynTypeM}[3]{\ensuremath{\withCtxM{#1}{\notSynTypeM{#2}{#3}}}}

\newcommand{\anaTypeM}[2]{\ensuremath{#1 \goodcolor{\colorMText}{\Leftarrow} #2}}
\newcommand{\notAnaTypeM}[2]{\ensuremath{#1 \goodcolor{\colorFailSideJudge}{\not\Leftarrow} #2}}
\newcommand{\ctxAnaTypeM}[3]{
  \begin{highlightbox}{\colorMBkgAna}
    \ensuremath{\withCtxM{#1}{\anaTypeM{#2}{#3}}}
  \end{highlightbox}}
\newcommand{\ctxNotAnaTypeM}[3]{\ensuremath{\withCtxM{#1}{\notAnaTypeM{#2}{#3}}}}

% marking
\newcommand{\colorMKText}{Bittersweet}
\newcommand{\colorMKBkgSyn}{Gray!5}
\newcommand{\colorMKBkgAna}{Gray!5}

\newcommand{\withCtxMK}[2]{\ensuremath{#1 \goodcolor{\colorMKText}{\vdash} #2}}
\newcommand{\synTypeMK}[2]{\ensuremath{#1 \goodcolor{\colorMKText}{\Rightarrow} #2}}
\newcommand{\anaTypeMK}[2]{\ensuremath{#1 \goodcolor{\colorMKText}{\Leftarrow} #2}}
\newcommand{\synFixedInto}[3]{\ensuremath{\synTypeMK{#1 \goodcolor{\colorMKText}{\looparrowright} #2}{#3}}}
\newcommand{\ctxSynFixedInto}[4]{
  \begin{highlightbox}{\colorMKBkgSyn}
    \ensuremath{{\withCtxMK{#1}{\synFixedInto{#2}{#3}{#4}}}}
  \end{highlightbox}}
\newcommand{\anaFixedInto}[3]{\ensuremath{\anaTypeMK{#1 \goodcolor{\colorMKText}{\looparrowright} #2}{#3}}}
\newcommand{\ctxAnaFixedInto}[4]{
  \begin{highlightbox}{\colorMKBkgAna}
    \ensuremath{\withCtxMK{#1}{\anaFixedInto{#2}{#3}{#4}}}
  \end{highlightbox}}

%
% judgments
%

% subsumable
\newcommand{\subsumable}[1]{\ensuremath{#1 ~{\normalfont\textsf{subsumable}}}}

% markless
\newcommand{\markless}[1]{\ensuremath{#1 ~{\normalfont\textsf{markless}}}}

% mark erasure
\newcommand{\erase}[1]{\ensuremath{#1^{\square}}}
\newcommand{\erasesTo}[2]{\ensuremath{#1^{\square} = #2}}

% !requires types, marked

%
% patterns
%

% unmarked
\newcommand{\PMName}{{\normalfont\textsf{UPat}}}
\newcommand{\PMV}{\ensuremath{p}}

\newcommand{\PWild}{\ensuremath{\PCWild}}
\newcommand{\PVar}[1]{\ensuremath{\PCVar{#1}}}
\newcommand{\PAsc}[2]{\ensuremath{\PCAsc{#1}{#2}}}
\newcommand{\PPair}[2]{\ensuremath{\PCPair{#1}{#2}}}

% marked
\newcommand{\PCMName}{{\normalfont\textsf{MPat}}}
\newcommand{\PCMV}{\ensuremath{\check{p}}}

\newcommand{\PCWild}{\ensuremath{\_}}
\newcommand{\PCVar}[1]{\ensuremath{#1}}
\newcommand{\PCAsc}[2]{\ensuremath{\assignType{#1}{#2}}}
\newcommand{\PCPair}[2]{\ensuremath{\ECPair{#1}{#2}}}
\newcommand{\PCPairAnaNonMatchedProd}[2]{\ensuremath{\ECAnaNonMatchedProd{\PCPair{#1}{#2}}}}
\newcommand{\PCInconType}[1]{\ensuremath{\ECInconType{#1}}}

\newcommand{\PCAnaNonMatchedProd}[1]{\ensuremath{\ECAnaNonMatchedProd{#1}}}

%
% typing
%

% unmarked
\newcommand{\ctxSynPatU}[3]{
  \begin{highlightbox}{\colorUBkgSyn}
    \ensuremath{\withCtxU{#1}{\synTypeU{#2}{#3}}}
  \end{highlightbox}}
\newcommand{\ctxAnaPatU}[4]{
  \begin{highlightbox}{\colorUBkgAna}
    \ensuremath{\withCtxU{#1}{\anaTypeU{#2}{#3 \goodcolor{\colorUText}{\dashv} #4}}}
  \end{highlightbox}}
\newcommand{\notAnaPatU}[2]{\ensuremath{#1 \goodcolor{\colorFailSideJudge}{\not\Leftarrow} #2}}
\newcommand{\ctxNotAnaPatU}[4]{
  \begin{highlightbox}{\colorUBkgAna}
    \ensuremath{\withCtxU{#1}{\notAnaPatU{#2}{#3 \goodcolor{\colorUText}{\dashv} #4}}}
  \end{highlightbox}}

% marked
\newcommand{\ctxSynPatM}[3]{
  \begin{highlightbox}{\colorMBkgSyn}
    \ensuremath{\withCtxM{#1}{\synTypeM{#2}{#3}}}
  \end{highlightbox}}
\newcommand{\ctxAnaPatM}[4]{
  \begin{highlightbox}{\colorMBkgAna}
    \ensuremath{\withCtxM{#1}{\anaTypeM{#2}{#3 \dashv #4}}}
  \end{highlightbox}}
\newcommand{\notAnaPatM}[2]{\ensuremath{#1 \goodcolor{\colorFailSideJudge}{\not\Leftarrow} #2}}
\newcommand{\ctxNotAnaPatM}[4]{\ensuremath{\withCtxM{#1}{\notAnaPatM{#2}{#3 \dashv #4}}}}

% marking
\newcommand{\withCtxMKPat}[2]{\ensuremath{#1 \goodcolor{\colorMKText}{\vdash} #2}}
\newcommand{\synTypeMKPat}[2]{\ensuremath{#1 \goodcolor{\colorMKText}{\Rightarrow} #2}}
\newcommand{\anaTypeMKPAt}[2]{\ensuremath{#1 \goodcolor{\colorMKText}{\Leftarrow} #2}}
\newcommand{\synFixedIntoPat}[3]{\ensuremath{\synTypeMKPat{#1 \goodcolor{\colorMKText}{\looparrowright} #2}{#3}}}
\newcommand{\ctxSynFixedIntoPat}[4]{
  \begin{highlightbox}{\colorMKBkgSyn}
    \ensuremath{\withCtxMKPat{#1}{\synFixedIntoPat{#2}{#3}{#4}}}
  \end{highlightbox}}
\newcommand{\anaFixedIntoPat}[3]{\ensuremath{\anaTypeMKPAt{#1 \goodcolor{\colorMKText}{\looparrowright} #2}{#3}}}
\newcommand{\ctxAnaFixedIntoPat}[5]{
  \begin{highlightbox}{\colorMKBkgAna}
    \ensuremath{\withCtxMKPat{#1}{\anaFixedIntoPat{#2}{#3}{#4 \dashv #5}}}
  \end{highlightbox}}

% !requires types, marked

%
% types
%

% syntax
\newcommand{\TVarMV}{\ensuremath{\alpha}}
\newcommand{\TForall}[2]{\ensuremath{\forall #1. ~#2}}

% substitution
\newcommand{\subst}[3]{\ensuremath{#1[#2 / #3]}}

% marked
\newcommand{\MTMName}{{\normalfont\textsf{MType}}}
\newcommand{\MTMV}{\ensuremath{\check{\TMV}}}

\newcommand{\MTMeet}[2]{\ensuremath{\TMeet{#1}{#2}}}

\newcommand{\MTUnknown}{\ensuremath{\TUnknown}}
\newcommand{\MTNum}{\ensuremath{\TNum}}
\newcommand{\MTBool}{\ensuremath{\TBool}}
\newcommand{\MTArrow}[2]{\ensuremath{\TArrow{#1}{#2}}}
\newcommand{\MTProd}[2]{\ensuremath{\TProd{#1}{#2}}}
\newcommand{\MTVarMV}{\ensuremath{\TVarMV}}
\newcommand{\MTFree}[1]{\ensuremath{\MTMarkedFree{#1}}}
\newcommand{\MTForall}[2]{\ensuremath{\TForall{#1}{#2}}}

\newcommand{\MTMarked}[3]{\ensuremath{\textcolor{red}{\bm{\llparenthesis}#1\bm{\rrparenthesis}_{#2}^{#3}}}}
\newcommand{\MTMarkedFree}[1]{\ensuremath{\MTMarked{#1}{\MRFree}{}}}

%
% expressions
%

% unmarked
\newcommand{\ETypeLam}[2]{\ensuremath{\ECTypeLam{#1}{#2}}}
\newcommand{\ETypeAp}[2]{\ensuremath{\ECTypeAp{#1}{#2}}}

% marked
\newcommand{\ECTypeLam}[2]{\ensuremath{\Lambda #1. ~#2}}
\newcommand{\ECTypeAp}[2]{\ensuremath{#1 ~[#2]}}

\newcommand{\MRSynNonMatchedForall}{\ensuremath{\mathsmaller{\notMatchedForallRel}}}
\newcommand{\MRAnaNonMatchedForall}{\ensuremath{\mathsmaller{\notMatchedForallRel}}}

\newcommand{\ECMarkedSynNonMatchedForall}[1]{\ensuremath{\ECMarked{#1}{\MRSynNonMatchedForall}{\MRSyn}}}
\newcommand{\ECMarkedAnaNonMatchedForall}[1]{\ensuremath{\ECMarked{#1}{\MRAnaNonMatchedForall}{\MRAna}}}

\newcommand{\ECSynNonMatchedForall}[1]{\ensuremath{\ECMarkedSynNonMatchedForall{#1}}}
\newcommand{\ECAnaNonMatchedForall}[1]{\ensuremath{\ECMarkedAnaNonMatchedForall{#1}}}

\newcommand{\ECTypeApSynNonMatchedForall}[2]{\ensuremath{\ECTypeAp{\ECSynNonMatchedForall{#1}}{#2}}}
\newcommand{\ECTypeLamAnaNonMatchedForall}[2]{\ensuremath{\ECAnaNonMatchedForall{\ECTypeLam{#1}{#2}}}}

%
% contexts
%
\newcommand{\tvarCtx}{\ensuremath{\Sigma}}
\newcommand{\extendTvarCtx}[2]{\ensuremath{#1, #2}}
\newcommand{\inTvarCtx}[2]{\ensuremath{#2 \in #1}}
\newcommand{\notInTvarCtx}[2]{\ensuremath{\goodcolor{\colorFailSideJudge}{#2 \not\in #1}}}
\newcommand{\withTvarCtx}[2]{\ensuremath{#1 \vdash #2}}
\newcommand{\withTvarCtxNot}[2]{\ensuremath{\goodcolor{\colorFailSideJudge}{#1 \centernot\vdash #2}}}
\newcommand{\withTvarCtxU}[2]{\ensuremath{#1 \goodcolor{\colorUText}{\vdash_{\hspace{-4.25pt}\scaleto{U}{3.5pt}}} #2}}
\newcommand{\withTvarCtxUNot}[2]{\ensuremath{\goodcolor{\colorFailSideJudge}{#1 \centernot\vdash_{\hspace{-4.25pt}\scaleto{U}{3.5pt}} #2}}}
\newcommand{\withTvarCtxM}[2]{\ensuremath{#1 \goodcolor{\colorMText}{\vdash_{\hspace{-4.25pt}\scaleto{M}{3.5pt}}} #2}}
\newcommand{\withTvarCtxMNot}[2]{\ensuremath{\goodcolor{\colorFailSideJudge}{#1 \centernot\vdash_{\hspace{-4.25pt}\scaleto{M}{3.5pt}} #2}}}
\newcommand{\withTvarCtxMK}[2]{\ensuremath{#1 \goodcolor{\colorMKText}{\vdash} #2}}

\newcommand{\withBothCtx}[3]{\ensuremath{#1; #2 \vdash #3}}
\newcommand{\withBothCtxU}[3]{\ensuremath{#1; #2 \goodcolor{\colorUText}{\vdash_{\hspace{-4.25pt}\scaleto{U}{3.5pt}}} #3}}
\newcommand{\withBothCtxM}[3]{\ensuremath{#1; #2 \goodcolor{\colorMText}{\vdash_{\hspace{-4.75pt}\scaleto{M}{3.5pt}}} #3}}
\newcommand{\withBothCtxMK}[3]{\ensuremath{#1; #2 \goodcolor{\colorMKText}{\vdash} #3}}

%
% judgments
%

% consistency
\newcommand{\tvarCtxConsistent}[3]{\ensuremath{\withTvarCtx{#1}{\consistent{#2}{#3}}}}
\newcommand{\tvarCtxConsistentU}[3]{\ensuremath{\withTvarCtxU{#1}{\consistent{#2}{#3}}}}
\newcommand{\tvarCtxConsistentM}[3]{\ensuremath{\withTvarCtxM{#1}{\consistent{#2}{#3}}}}

% well-formedness
\newcommand{\tvarCtxWF}[2]{\ensuremath{\withTvarCtx{#1}{#2}}}
\newcommand{\tvarCtxWFU}[2]{
  \begin{highlightbox}{\colorUBkgSyn}
    \ensuremath{\withTvarCtxU{#1}{#2}}
  \end{highlightbox}}
\newcommand{\tvarCtxWFM}[2]{
  \begin{highlightbox}{\colorMBkgSyn}
    \ensuremath{\withTvarCtxM{#1}{#2}}
  \end{highlightbox}}

\newcommand{\tvarCtxNotWF}[2]{\ensuremath{\withTvarCtxNot{#1}{#2}}}
\newcommand{\tvarCtxNotWFU}[2]{\ensuremath{\withTvarCtxUNot{#1}{#2}}}
\newcommand{\tvarCtxNotWFM}[2]{\ensuremath{\withTvarCtxMNot{#1}{#2}}}

% matched forall
\newcommand{\matchedForallRel}{\ensuremath{\matchedRel{\forall}}}
\newcommand{\notMatchedForallRel}{\ensuremath{\notMatchedRel{\forall}}}
\newcommand{\matchedForall}[3]{\ensuremath{#1 \goodcolor{\colorOkSideJudge}{\matchedForallRel} \TForall{#2}{#3}}}
\newcommand{\notMatchedForall}[1]{\ensuremath{\goodcolor{\colorFailSideJudge}{#1 \notMatchedForallRel}}}

% type marking
\newcommand{\typeMarkedInto}[2]{\ensuremath{#1 \goodcolor{\colorMKText}{\looparrowright} #2}}
\newcommand{\tvarCtxTypeMarkedInto}[3]{
  \begin{highlightbox}{\colorMKBkgSyn}
    \ensuremath{{\withTvarCtxMK{#1}{\typeMarkedInto{#2}{#3}}}}
  \end{highlightbox}}

% unmarked typing
\newcommand{\bothCtxSynTypeU}[4]{
  \begin{highlightbox}{\colorUBkgSyn}
    \ensuremath{\withBothCtxU{#1}{#2}{\synTypeU{#3}{#4}}}
  \end{highlightbox}}
\newcommand{\bothCtxAnaTypeU}[4]{
  \begin{highlightbox}{\colorUBkgAna}
    \ensuremath{\withBothCtxU{#1}{#2}{\anaTypeU{#3}{#4}}}
  \end{highlightbox}}

% marked typing
\newcommand{\bothCtxSynTypeM}[4]{
  \begin{highlightbox}{\colorMBkgSyn}
    \ensuremath{\withBothCtxM{#1}{#2}{\synTypeM{#3}{#4}}}
  \end{highlightbox}}
\newcommand{\bothCtxAnaTypeM}[4]{
  \begin{highlightbox}{\colorMBkgAna}
    \ensuremath{\withBothCtxM{#1}{#2}{\anaTypeM{#3}{#4}}}
  \end{highlightbox}}

% marking
\newcommand{\bothCtxSynFixedInto}[5]{
  \begin{highlightbox}{\colorMKBkgSyn}
    \ensuremath{{\withBothCtxMK{#1}{#2}{\synFixedInto{#3}{#4}{#5}}}}
  \end{highlightbox}}
\newcommand{\bothCtxAnaFixedInto}[5]{
  \begin{highlightbox}{\colorMKBkgAna}
    \ensuremath{\withBothCtxMK{#1}{#2}{\anaFixedInto{#3}{#4}{#5}}}
  \end{highlightbox}}

% !requires types

%
% zippered types
%
\newcommand{\ZTMName}{{\normalfont\textsf{ZType}}}
\newcommand{\ZTMSet}{\ensuremath{\hat{T}}}
\newcommand{\ZTMV}{\ensuremath{\hat{\tau}}}

% cursor
\newcommand{\ZTCursor}[1]{\ensuremath{\ZCursor{#1}}}

% arrow
\newcommand{\ZTArrowL}[2]{\ensuremath{\TArrow{#1}{#2}}}
\newcommand{\ZTArrowR}[2]{\ensuremath{\TArrow{#1}{#2}}}
\newcommand{\ZTProdL}[2]{\ensuremath{\TProd{#1}{#2}}}
\newcommand{\ZTProdR}[2]{\ensuremath{\TProd{#1}{#2}}}

%
% zippered expressions
%
\newcommand{\ZMName}{{\normalfont\textsf{ZExp}}}
\newcommand{\ZMSet}{\ensuremath{\hat{E}}}
\newcommand{\ZMV}{\ensuremath{\hat{e}}}

% cursor
\definecolor{cursorhighlight}{RGB}{230,255,230}
\definecolor{cursor}{RGB}{76,170,76}
\newcommand{\ZCursor}[1]{\ensuremath{\mathcolorbox{cursorhighlight}{\textcolor{cursor}{\bm{\triangleright}}#1\textcolor{cursor}{\bm{\triangleleft}}}}}

% integers
\newcommand{\ZPlusL}[2]{\ensuremath{\ECPlus{#1}{#2}}}
\newcommand{\ZPlusR}[2]{\ensuremath{\ECPlus{#1}{#2}}}

% lambdas
\newcommand{\ZLamT}[3]{\ensuremath{\ECLam{#1}{#2}{#3}}}
\newcommand{\ZLamE}[3]{\ensuremath{\ECLam{#1}{#2}{#3}}}
\newcommand{\ZApL}[2]{\ensuremath{\ECAp{#1}{#2}}}
\newcommand{\ZApR}[2]{\ensuremath{\ECAp{#1}{#2}}}
\newcommand{\ZApNonMatchedL}[2]{\ensuremath{\ECApNonMatched{#1}{#2}}}
\newcommand{\ZApNonMatchedR}[2]{\ensuremath{\ECApNonMatched{#1}{#2}}}

% booleans
\newcommand{\ZIfC}[3]{\ensuremath{\ECIf{#1}{#2}{#3}}}
\newcommand{\ZIfL}[3]{\ensuremath{\ECIf{#1}{#2}{#3}}}
\newcommand{\ZIfR}[3]{\ensuremath{\ECIf{#1}{#2}{#3}}}
\newcommand{\ZInconBrC}[3]{\ensuremath{\ECInconBr{#1}{#2}{#3}}}
\newcommand{\ZInconBrL}[3]{\ensuremath{\ECInconBr{#1}{#2}{#3}}}
\newcommand{\ZInconBrR}[3]{\ensuremath{\ECInconBr{#1}{#2}{#3}}}

% let
\newcommand{\ZLetL}[3]{\ensuremath{\ECLet{#1}{#2}{#3}}}
\newcommand{\ZLetR}[3]{\ensuremath{\ECLet{#1}{#2}{#3}}}

% pairs
\newcommand{\ZPairL}[2]{\ensuremath{\ECPair{#1}{#2}}}
\newcommand{\ZPairR}[2]{\ensuremath{\ECPair{#1}{#2}}}
\newcommand{\ZProjL}[1]{\ensuremath{\ECProjL{#1}}}
\newcommand{\ZProjR}[1]{\ensuremath{\ECProjR{#1}}}

% errors
\newcommand{\ZFree}[1]{\ensuremath{\ECFree{#1}}}
\newcommand{\ZInconType}[1]{\ensuremath{\ECInconType{#1}}}
\newcommand{\ZInconBr}[3]{\ensuremath{\ECInconBr{#1}{#2}{#3}}}

%
% zipper judgments
%

% cursor erasure
\newcommand{\cursorErase}[1]{\ensuremath{#1^{\diamond}}}
\newcommand{\cursorErasesTo}[2]{\ensuremath{#1^{\diamond} = #2}}

% movements
\newcommand{\movements}[1]{\ensuremath{#1 ~{\normalfont\textsf{movements}}}}

% shape
\newcommand{\tshape}[1]{\ensuremath{#1 ~{\normalfont\textsf{tshape}}}}
\newcommand{\eshape}[1]{\ensuremath{#1 ~{\normalfont\textsf{eshape}}}}

%
% actions
%
\newcommand{\AMName}{{\normalfont\textsf{Action}}}
\newcommand{\AMV}{\ensuremath{\alpha}}
\newcommand{\AMove}[1]{\ensuremath{{\normalfont\textsf{move $#1$}}}}
\newcommand{\ADel}{\ensuremath{{\normalfont\textsf{del}}}}
\newcommand{\ACon}[1]{\ensuremath{{\normalfont\textsf{construct $#1$}}}}

%
% iterated actions
%
\newcommand{\AIMName}{{\normalfont\textsf{ActionList}}}
\newcommand{\AIMV}{\ensuremath{\overline{\alpha}}}
\newcommand{\AINil}{\ensuremath{\cdot}}
\newcommand{\AICons}[2]{\ensuremath{#1; #2}}

%
% directions
%
\newcommand{\MMName}{{\normalfont\textsf{Dir}}}
\newcommand{\MMV}{\ensuremath{\delta}}
\newcommand{\MChild}[1]{\ensuremath{{\normalfont\textsf{child $#1$}}}}
\newcommand{\MChildNMV}{\ensuremath{n}}
\newcommand{\MParent}{\ensuremath{{\normalfont\textsf{parent}}}}

%
% shapes
%
\newcommand{\SMName}{{\normalfont\textsf{Shape}}}
\newcommand{\SMV}{\ensuremath{\psi}}

\newcommand{\STArrowL}{\ensuremath{{\normalfont\textsf{arrow\textsubscript{L}}}}}
\newcommand{\STArrowR}{\ensuremath{{\normalfont\textsf{arrow\textsubscript{R}}}}}
\newcommand{\STProdL}{\ensuremath{{\normalfont\textsf{prod\textsubscript{L}}}}}
\newcommand{\STProdR}{\ensuremath{{\normalfont\textsf{prod\textsubscript{R}}}}}
\newcommand{\STNum}{\ensuremath{{\normalfont\textsf{num}}}}
\newcommand{\STBool}{\ensuremath{{\normalfont\textsf{bool}}}}

\newcommand{\SVar}[1]{\ensuremath{{\normalfont\textsf{var $#1$}}}}
\newcommand{\SLam}[1]{\ensuremath{{\normalfont\textsf{lam $#1$}}}}
\newcommand{\SApL}{\ensuremath{{\normalfont\textsf{ap\textsubscript{L}}}}}
\newcommand{\SApR}{\ensuremath{{\normalfont\textsf{ap\textsubscript{R}}}}}
\newcommand{\SLetL}[1]{\ensuremath{{\normalfont\textsf{let\textsubscript{L} $#1$}}}}
\newcommand{\SLetR}[1]{\ensuremath{{\normalfont\textsf{let\textsubscript{R} $#1$}}}}
\newcommand{\SLit}[1]{\ensuremath{{\normalfont\textsf{lit $#1$}}}}
\newcommand{\SPlusL}{\ensuremath{{\normalfont\textsf{plus\textsubscript{L}}}}}
\newcommand{\SPlusR}{\ensuremath{{\normalfont\textsf{plus\textsubscript{R}}}}}
\newcommand{\STrue}{\ensuremath{{\normalfont\textsf{true}}}}
\newcommand{\SFalse}{\ensuremath{{\normalfont\textsf{false}}}}
\newcommand{\SIfC}{\ensuremath{{\normalfont\textsf{if\textsubscript{C}}}}}
\newcommand{\SIfL}{\ensuremath{{\normalfont\textsf{if\textsubscript{L}}}}}
\newcommand{\SIfR}{\ensuremath{{\normalfont\textsf{if\textsubscript{R}}}}}
\newcommand{\SPairL}{\ensuremath{{\normalfont\textsf{pair\textsubscript{L}}}}}
\newcommand{\SPairR}{\ensuremath{{\normalfont\textsf{pair\textsubscript{R}}}}}
\newcommand{\SProjL}{\ensuremath{{\normalfont\textsf{proj\textsubscript{L}}}}}
\newcommand{\SProjR}{\ensuremath{{\normalfont\textsf{proj\textsubscript{R}}}}}

%
% action judgments
%
\newcommand{\AUAction}[3]{\ensuremath{#1 \xrightarrow{#3} #2}}
\newcommand{\AUActionIter}[3]{\ensuremath{#1 \xrightarrow{#3}* #2}}

% movement
\newcommand{\AUMChild}[3]{\ensuremath{\AUAction{#1}{#2}{\AMove{\MChild{#3}}}}}
\newcommand{\AUMParent}[2]{\ensuremath{\AUAction{#1}{#2}{\AMove{\MParent}}}}

% deletion
\newcommand{\AUDel}[2]{\ensuremath{\AUAction{#1}{#2}{\ADel}}}

% construction
\newcommand{\AUCon}[3]{\ensuremath{\AUAction{#1}{#2}{\ACon{#3}}}}

%
% type actions
%
\newcommand{\AUTAction}[3]{\ensuremath{\AUAction{#1}{#2}{#3}}}
\newcommand{\AUTActionIter}[3]{\ensuremath{\AUActionIter{#1}{#2}{#3}}}

% movement
\newcommand{\AUTMChild}[3]{\ensuremath{\AUMChild{\ZTCursor{#1}}{#2}{#3}}}
\newcommand{\AUTMParent}[2]{\ensuremath{\AUMParent{#1}{\ZTCursor{#2}}}}
\newcommand{\AUTMArrowChildL}[2]{\ensuremath{\AUTMChild{\TArrow{#1}{#2}}{\ZTArrowL{\ZTCursor{#1}}{#2}}{1}}}
\newcommand{\AUTMArrowChildR}[2]{\ensuremath{\AUTMChild{\TArrow{#1}{#2}}{\ZTArrowR{#2}{\ZTCursor{#1}}}{2}}}
\newcommand{\AUTMArrowParentL}[2]{\ensuremath{\AUTMParent{\ZTArrowL{\ZTCursor{#1}}{#2}}{\TArrow{#1}{#2}}}}
\newcommand{\AUTMArrowParentR}[2]{\ensuremath{\AUTMParent{\ZTArrowR{#2}{\ZTCursor{#1}}}{\TArrow{#1}{#2}}}}
\newcommand{\AUTMProdChildL}[2]{\ensuremath{\AUTMChild{\TProd{#1}{#2}}{\ZTProdL{\ZTCursor{#1}}{#2}}{1}}}
\newcommand{\AUTMProdChildR}[2]{\ensuremath{\AUTMChild{\TProd{#1}{#2}}{\ZTProdR{#2}{\ZTCursor{#1}}}{2}}}
\newcommand{\AUTMProdParentL}[2]{\ensuremath{\AUTMParent{\ZTProdL{\ZTCursor{#1}}{#2}}{\TProd{#1}{#2}}}}
\newcommand{\AUTMProdParentR}[2]{\ensuremath{\AUTMParent{\ZTProdR{#2}{\ZTCursor{#1}}}{\TProd{#1}{#2}}}}

% deletion
\newcommand{\AUTDel}[1]{\ensuremath{\AUDel{\ZTCursor{#1}}{\ZTCursor{\TUnknown}}}}

% construction
\newcommand{\AUTConArrowL}[1]{\ensuremath{\AUCon{\ZTCursor{#1}}{\ZTArrowR{#1}{\ZTCursor{\TUnknown}}}{\STArrowL}}}
\newcommand{\AUTConArrowR}[1]{\ensuremath{\AUCon{\ZTCursor{#1}}{\ZTArrowR{\ZTCursor{\TUnknown}}{#1}}{\STArrowR}}}
\newcommand{\AUTConProdL}[1]{\ensuremath{\AUCon{\ZTCursor{#1}}{\ZTProdR{#1}{\ZTCursor{\TUnknown}}}{\STProdL}}}
\newcommand{\AUTConProdR}[1]{\ensuremath{\AUCon{\ZTCursor{#1}}{\ZTProdR{\ZTCursor{\TUnknown}}{#1}}{\STProdR}}}
\newcommand{\AUTConNum}{\ensuremath{\AUCon{\ZTCursor{\TUnknown}}{\ZTCursor{\TNum}}{\STNum}}}
\newcommand{\AUTConBool}{\ensuremath{\AUCon{\ZTCursor{\TUnknown}}{\ZTCursor{\TBool}}{\STBool}}}

%
% untyped expression actions
%
\newcommand{\AUEAction}[3]{\ensuremath{\AUAction{#1}{#2}{#3}}}
\newcommand{\AUEActionIter}[3]{\ensuremath{\AUActionIter{#1}{#2}{#3}}}

% movement
\newcommand{\AUEMove}[3]{\ensuremath{\AUEAction{#1}{#2}{\AMove{\MMV}}}}
\newcommand{\AUEMChild}[3]{\ensuremath{\AUMChild{\ZCursor{#1}}{#2}{#3}}}
\newcommand{\AUEMParent}[2]{\ensuremath{\AUMParent{#1}{\ZCursor{#2}}}}

\newcommand{\AUEMLamChildT}[3]{\ensuremath{\AUEMChild{\ELam{#1}{#2}{#3}}{\ZLamT{#1}{\ZTCursor{#2}}{#3}}{1}}}
\newcommand{\AUEMLamChildE}[3]{\ensuremath{\AUEMChild{\ELam{#1}{#2}{#3}}{\ZLamE{#1}{#2}{\ZCursor{#3}}}{2}}}
\newcommand{\AUEMLamParentT}[3]{\ensuremath{\AUEMParent{\ZLamT{#1}{\ZTCursor{#2}}{#3}}{\ELam{#1}{#2}{#3}}}}
\newcommand{\AUEMLamParentE}[3]{\ensuremath{\AUEMParent{\ZLamE{#1}{#2}{\ZCursor{#3}}}{\ELam{#1}{#2}{#3}}}}
\newcommand{\AUEMApChildL}[2]{\ensuremath{\AUEMChild{\EAp{#1}{#2}}{\ZApL{\ZCursor{#1}}{#2}}{1}}}
\newcommand{\AUEMApChildR}[2]{\ensuremath{\AUEMChild{\EAp{#1}{#2}}{\ZApR{#1}{\ZCursor{#2}}}{2}}}
\newcommand{\AUEMApParentL}[2]{\ensuremath{\AUEMParent{\ZApL{\ZCursor{#1}}{#2}}{\EAp{#1}{#2}}}}
\newcommand{\AUEMApParentR}[2]{\ensuremath{\AUEMParent{\ZApR{#1}{\ZCursor{#2}}}{\EAp{#1}{#2}}}}
\newcommand{\AUEMLetChildL}[3]{\ensuremath{\AUEMChild{\ELet{#1}{#2}{#3}}{\ZLetL{#1}{\ZCursor{#2}}{#3}}{1}}}
\newcommand{\AUEMLetChildR}[3]{\ensuremath{\AUEMChild{\ELet{#1}{#2}{#3}}{\ZLetR{#1}{#2}{\ZCursor{#3}}}{2}}}
\newcommand{\AUEMLetParentL}[3]{\ensuremath{\AUEMParent{\ZLetL{#1}{\ZCursor{#2}}{#3}}{\ELet{#1}{#2}{#3}}}}
\newcommand{\AUEMLetParentR}[3]{\ensuremath{\AUEMParent{\ZLetR{#1}{#2}{\ZCursor{#3}}}{\ELet{#1}{#2}{#3}}}}
\newcommand{\AUEMPlusChildL}[2]{\ensuremath{\AUEMChild{\EPlus{#1}{#2}}{\ZPlusL{\ZCursor{#1}}{#2}}{1}}}
\newcommand{\AUEMPlusChildR}[2]{\ensuremath{\AUEMChild{\EPlus{#1}{#2}}{\ZPlusR{#1}{\ZCursor{#2}}}{2}}}
\newcommand{\AUEMPlusParentL}[2]{\ensuremath{\AUEMParent{\ZPlusL{\ZCursor{#1}}{#2}}{\EPlus{#1}{#2}}}}
\newcommand{\AUEMPlusParentR}[2]{\ensuremath{\AUEMParent{\ZPlusR{#1}{\ZCursor{#2}}}{\EPlus{#1}{#2}}}}
\newcommand{\AUEMIfChildC}[3]{\ensuremath{\AUEMChild{\EIf{#1}{#2}{#3}}{\ZIfC{\ZCursor{#1}}{#2}{#3}}{1}}}
\newcommand{\AUEMIfChildL}[3]{\ensuremath{\AUEMChild{\EIf{#1}{#2}{#3}}{\ZIfL{#1}{\ZCursor{#2}}{#3}}{2}}}
\newcommand{\AUEMIfChildR}[3]{\ensuremath{\AUEMChild{\EIf{#1}{#2}{#3}}{\ZIfR{#1}{#2}{\ZCursor{#3}}}{3}}}
\newcommand{\AUEMIfParentC}[3]{\ensuremath{\AUEMParent{\ZIfC{\ZCursor{#1}}{#2}{#3}}{\EIf{#1}{#2}{#3}}}}
\newcommand{\AUEMIfParentL}[3]{\ensuremath{\AUEMParent{\ZIfL{#1}{\ZCursor{#2}}{#3}}{\EIf{#1}{#2}{#3}}}}
\newcommand{\AUEMIfParentR}[3]{\ensuremath{\AUEMParent{\ZIfR{#1}{#2}{\ZCursor{#3}}}{\EIf{#1}{#2}{#3}}}}
\newcommand{\AUEMPairChildL}[2]{\ensuremath{\AUEMChild{\EPair{#1}{#2}}{\ZPairL{\ZCursor{#1}}{#2}}{1}}}
\newcommand{\AUEMPairChildR}[2]{\ensuremath{\AUEMChild{\EPair{#1}{#2}}{\ZPairR{#1}{\ZCursor{#2}}}{2}}}
\newcommand{\AUEMPairParentL}[2]{\ensuremath{\AUEMParent{\ZPairL{\ZCursor{#1}}{#2}}{\EPair{#1}{#2}}}}
\newcommand{\AUEMPairParentR}[2]{\ensuremath{\AUEMParent{\ZPairR{#1}{\ZCursor{#2}}}{\EPair{#1}{#2}}}}
\newcommand{\AUEMProjLChild}[1]{\ensuremath{\AUEMChild{\EProjL{#1}}{\ZProjL{\ZCursor{#1}}}{1}}}
\newcommand{\AUEMProjLParent}[1]{\ensuremath{\AUEMParent{\ZProjL{\ZCursor{#1}}}{\EProjL{#1}}}}
\newcommand{\AUEMProjRChild}[1]{\ensuremath{\AUEMChild{\EProjR{#1}}{\ZProjR{\ZCursor{#1}}}{1}}}
\newcommand{\AUEMProjRParent}[1]{\ensuremath{\AUEMParent{\ZProjR{\ZCursor{#1}}}{\EProjR{#1}}}}

% deletion
\newcommand{\AUEDel}[1]{\ensuremath{\AUDel{\ZCursor{#1}}{\ZCursor{\EEHole}}}}

% construction
\newcommand{\AUEConVar}[1]{\ensuremath{\AUCon{\ZCursor{\EEHole}}{\ZCursor{#1}}{\SVar{#1}}}}
\newcommand{\AUEConLam}[2]{\ensuremath{\AUCon{\ZCursor{#2}}{\ZLamT{#1}{\ZTCursor{\TUnknown}}{#2}}{\SLam{#1}}}}
\newcommand{\AUEConApL}[1]{\ensuremath{\AUCon{\ZCursor{#1}}{\ZApR{#1}{\ZCursor{\EEHole}}}{\SApL}}}
\newcommand{\AUEConApR}[1]{\ensuremath{\AUCon{\ZCursor{#1}}{\ZApR{\ZCursor{\EEHole}}{#1}}{\SApR}}}
\newcommand{\AUEConLetL}[2]{\ensuremath{\AUCon{\ZCursor{#2}}{\ZLetL{#1}{#2}{\ZCursor{\EEHole}}}{\SLetL{#1}}}}
\newcommand{\AUEConLetR}[2]{\ensuremath{\AUCon{\ZCursor{#2}}{\ZLetR{#1}{\ZCursor{\EEHole}}{#2}}{\SLetR{#1}}}}
\newcommand{\AUEConNum}[1]{\ensuremath{\AUCon{\ZCursor{\EEHole}}{\ZCursor{#1}}{\SLit{#1}}}}
\newcommand{\AUEConPlusL}[1]{\ensuremath{\AUCon{\ZCursor{#1}}{\ZPlusL{#1}{\ZCursor{\EEHole}}}{\SPlusL}}}
\newcommand{\AUEConPlusR}[1]{\ensuremath{\AUCon{\ZCursor{#1}}{\ZPlusR{\ZCursor{\EEHole}}{#1}}{\SPlusR}}}
\newcommand{\AUEConTrue}{\ensuremath{\AUCon{\ZCursor{\EEHole}}{\ZCursor{\ETrue}}{\STrue}}}
\newcommand{\AUEConFalse}{\ensuremath{\AUCon{\ZCursor{\EEHole}}{\ZCursor{\EFalse}}{\SFalse}}}
\newcommand{\AUEConIfC}[1]{\ensuremath{\AUCon{\ZCursor{#1}}{\ZIfC{#1}{\ZCursor{\EEHole}}{\EEHole}}{\SIfC}}}
\newcommand{\AUEConIfL}[1]{\ensuremath{\AUCon{\ZCursor{#1}}{\ZIfC{\ZCursor{\EEHole}}{#1}{\EEHole}}{\SIfC}}}
\newcommand{\AUEConIfR}[1]{\ensuremath{\AUCon{\ZCursor{#1}}{\ZIfC{\ZCursor{\EEHole}}{\EEHole}{#1}}{\SIfC}}}
\newcommand{\AUEConPairL}[1]{\ensuremath{\AUCon{\ZCursor{#1}}{\ZPairL{#1}{\ZCursor{\EEHole}}}{\SPairL}}}
\newcommand{\AUEConPairR}[1]{\ensuremath{\AUCon{\ZCursor{#1}}{\ZPairR{\ZCursor{\EEHole}}{#1}}{\SPairR}}}
\newcommand{\AUEConProjL}[1]{\ensuremath{\AUCon{\ZCursor{#1}}{\ZCursor{\ZProjL{#1}}}{\SProjL}}}
\newcommand{\AUEConProjR}[1]{\ensuremath{\AUCon{\ZCursor{#1}}{\ZCursor{\ZProjR{#1}}}{\SProjR}}}

% !requires untyped

% zippered expression well-formedness
\newcommand{\zWellFormed}[1]{\ensuremath{#1 ~{\normalfont\textsf{WF}}}}

%
% typed movement actions
%
\newcommand{\ASEMove}[3]{\ensuremath{\AUEMove{#1}{#2}{#3}}}
\newcommand{\ASEMChild}[3]{\ensuremath{\AUEMChild{#1}{#2}{#3}}}
\newcommand{\ASEMParent}[2]{\ensuremath{\AUEMParent{#1}{#2}}}

%
% synthetic action judgments
%
\newcommand{\ASAction}[6]{\ensuremath{\AUAction{\ctxSynType{#1}{#2}{#3}}{\synType{#4}{#5}}{#6}}}
\newcommand{\ASActionIter}[6]{\ensuremath{\AUActionIter{\ctxSynType{#1}{#2}{#3}}{\synType{#4}{#5}}{#6}}}

% movement
\newcommand{\ASMove}[6]{\ensuremath{\ASAction{#1}{#2}{#3}{#4}{#5}{\AMove{\MMV}}}}
\newcommand{\ASMChild}[6]{\ensuremath{\ASAction{#1}{#2}{#3}{#4}{#5}{\AMove{\MChild{#6}}}}}
\newcommand{\ASMParent}[5]{\ensuremath{\ASAction{#1}{#2}{#3}{#4}{#5}{\AMove{\MParent}}}}

% deletion
\newcommand{\ASDel}[5]{\ensuremath{\ASAction{#1}{#2}{#3}{#4}{#5}{\ADel}}}

% construction
\newcommand{\ASCon}[6]{\ensuremath{\ASAction{#1}{#2}{#3}{#4}{#5}{\ACon{#6}}}}

%
% synthetic expression actions
%
\newcommand{\ASEAction}[6]{\ensuremath{\ASAction{#1}{#2}{#3}{#4}{#5}{#6}}}
\newcommand{\ASEActionIter}[6]{\ensuremath{\ASActionIter{#1}{#2}{#3}{#4}{#5}{#6}}}

% deletion
\newcommand{\ASEDel}[3]{\ensuremath{\ASDel{#1}{\ZCursor{#2}}{#3}{\ZCursor{\ECEHole}}{\TUnknown}}}

%
% analytic action judgments
%
\newcommand{\AAAction}[5]{\ensuremath{\AUAction{\withCtx{#1}{#2}}{\anaType{#3}{#4}}{#5}}}
\newcommand{\AAActionIter}[5]{\ensuremath{\AUActionIter{\withCtx{#1}{#2}}{\anaType{#3}{#4}}{#5}}}

% movement
\newcommand{\AAMove}[5]{\ensuremath{\AAAction{#1}{#2}{#3}{#4}{\AMove{#5}}}}
\newcommand{\AAMChild}[5]{\ensuremath{\AAAction{#1}{#2}{#3}{#4}{\AMove{\MChild{#5}}}}}
\newcommand{\AAMParent}[4]{\ensuremath{\AAAction{#1}{#2}{#3}{#4}{\AMove{\MParent}}}}

% deletion
\newcommand{\AADel}[4]{\ensuremath{\AAAction{#1}{#2}{#3}{#4}{\ADel}}}

% construction
\newcommand{\AACon}[5]{\ensuremath{\AAAction{#1}{#2}{#3}{#4}{\ACon{#5}}}}

%
% analytic expression actions
%
\newcommand{\AAEAction}[5]{\ensuremath{\AAAction{#1}{#2}{#3}{#4}{#5}}}
\newcommand{\AAEActionIter}[5]{\ensuremath{\AAActionIter{#1}{#2}{#3}{#4}{#5}}}

% deletion
\newcommand{\AAEDel}[3]{\ensuremath{\AADel{#1}{\ZCursor{#2}}{\ZCursor{\ECEHole}}{#3}}}


%
% judgments
%

% judgments
\newcommand{\judgment}[3]{\inferrule[#3]{#1}{#2}}
\newcommand{\judgbox}[1]{\noindent \fbox{$#1$}}


% constraints
\newcommand{\constrain}[2]{{#1} \approx {#2}}
\newcommand{\constraintCons}[2]{\normalfont\textsf{cons}(#1, #2)}
\newcommand{\constraintNil}{[]}


\newcommand{\matchedArrowConstraint}[4]{
    \matchedArrow{#1}{#2}{#3}\goodcolor{\colorSideJudge}{~|~} {#4}
}
\newcommand{\matchedProdConstraint}[4]{
    \matchedProd{#1}{#2}{#3}\goodcolor{\colorSideJudge}{~|~} {#4}
}
\newcommand{\constraintTurn}{\goodcolor{\colorCText}{\vdash}}
\newcommand{\constraintBar}{\goodcolor{\colorCText}{~|~}}
\newcommand{\constraintSyn}{\goodcolor{\colorCText}{\Rightarrow}}
\newcommand{\constraintAna}{\goodcolor{\colorCText}{\Leftarrow}}
\newcommand{\synConstraint}[4]{
    \begin{mybox}{\colorCBkgSyn}
    \ensuremath{{#1} \constraintTurn {#2} \constraintSyn {#3} \constraintBar {#4}}
    \end{mybox}
}
\newcommand{\anaConstraint}[4]{
    \begin{mybox}{\colorCBkgAna}
    \ensuremath{{#1} \constraintTurn {#2} \constraintAna {#3} \constraintBar {#4}}
    \end{mybox}
}

% type incomplete
\newcommand{\incomplete}[1]{#1 ~ \normalfont\textsf{incomplete}}

% potential type
\newcommand{\PTUnknownVar}[1]{\ensuremath{#1}}
\newcommand{\PTNum}{\ensuremath{{\normalfont\textsf{N}}}}
\newcommand{\PTBool}{\ensuremath{{\normalfont\textsf{B}}}}
\newcommand{\PTArrow}[2]{\ensuremath{#1 \to #2}}

% potential types
\newcommand{\ptypCons}[2]{\normalfont\textsf{cons}(#1, #2)}
\newcommand{\ptypSingle}[1]{\normalfont\textsf{single}(#1)}

% potential type set
\newcommand{\ptsRep}[1]{\normalfont\textsf{rep}(#1)}
\newcommand{\ptsLead}[1]{\normalfont\textsf{lead}(#1)}

% PTGraph Elements and Membership
\newcommand{\ptypGraphNode}[2]{#1\normalfont\textsf{ : }#2}
\newcommand{\underGraph}[2]{#1~|~#2}
\DeclareMathOperator{\cod}{cod}
\newcommand{\isIn}[2]{#1 \in #2}
\newcommand{\isNotIn}[2]{#1 \notin #2}

% PTGraph Update
\newcommand{\ptypGraphUpdate}[4]{
    \underGraph{#1}{\normalfont\textsf{update}~#2:#3 ~\hookrightarrow~ #4}
}

% PTGraph add node
\newcommand{\ptypGraphAdd}[3]{
    \underGraph{#1}{\normalfont\textsf{add}~#2 ~\hookrightarrow~ #3}
}

% PTSet Leader
\newcommand{\leader}[4]{
    \underGraph{#1}{\normalfont\textsf{leader}~#2 \equiv \ptypGraphNode{#3}{#4}}
}

% Unions
\newcommand{\unionOf}[3]{#2 \cup_{#1} #3}

%   Type Union
\newcommand{\typUnion}[2]{\unionOf{\tau}{#1}{#2}}
\newcommand{\ptypGraphUnion}[4]{
    \underGraph{#1}{\typUnion{#2}{#3}} ~\hookrightarrow~ #4
}

% PTyps Union
\newcommand{\ptypsUnion}[2]{#1~\Cup~#2}
\newcommand{\specSubset}[3]{#1~\Subset_{#3}~#2}
\newcommand{\ptypesSubset}[2]{\specSubset{#1}{#2}{u}}
\newcommand{\ptypSetSubset}[3]{\underGraph{#1}{\specSubset{#2}{#3}{s}}}
\newcommand{\snapSubset}[2]{\specSubset{#1}{#2}{j}}
\newcommand{\snapElement}[2]{#1 ~\in_{h}~ #2}

% unification
\newcommand{\unify}[3]{\underGraph{#1}{#2~\amalg~#3}}

% Snapshot PTS
\newcommand{\ptsSnapshot}[3]{\underGraph{#1}{\normalfont\textsf{snapshot}_s~#2 \equiv #3}}

% Snapshot List
\newcommand{\listSnapshot}[3]{\underGraph{#1}{\normalfont\textsf{snapshot}_l~#2 \equiv #3}}

% Solution
\newcommand{\solution}[2]{\normalfont\textsf{solution}~#1 \equiv #2}

% List
\newcommand{\listCons}[2]{\normalfont\textsf{cons}(#1, #2)}
\newcommand{\listNil}{\normalfont\textsf{nil}}

% fresh
\newcommand{\fresh}[1]{#1~\normalfont\textsf{fresh}}