\section{Motivating Examples}
In this section, we motivate a formal calculus for bidirectional type error localization through a
number of examples.

% TODO Talk about idea that bidirectional typing makes error localization easy?

\subsection{Unbound Variables}
% A simple example of an error that may occur in programs is the use of an unbound variable.

\subsection{Inconsistent Types}
Often, the type an expression \emph{should} be is known. For example, in an expression such as
$\EPlus{\EMV_1}{\EMV_2}$, both $\EMV_1$ and $\EMV_2$ should have the type $\TNum$. In a
bidirectional system, $\EMV_1$ and $\EMV_2$ would be analysed against $\TNum$.

\subsection{Inconsistent Branches}

\section{Marking Calculus}
\label{sec:calculus}

The previous section introduced different error localization cases by example. Now, we formally
introduce a \emph{marking calculus} for bidirectional type error localization on a lambda calculus
extended with numbers, conditionals, and let bindings, given by a two-layer system:
%
\begin{itemize}
  \item \textbf{Unmarked expressions} ($\EMName$), which are the surface-level expression syntax.
    Types classify unmarked expressions according to a standard bidirectional gradual typing
    semantics. Note that a non-gradual system may be recovered straightforwardly by removing the
    unknown type and replacing type consistency with equality (discussed in
    \cref{sec:calculus-non-gradual}).

  \item \textbf{Marked expressions} ($\ECMName$), which mirror the structure of unmarked expressions
    but are extended with \textbf{error markings}. Every marked expression corresponds to an
    unmarked expression by \emph{mark erasure}.

  \item \textbf{Mark insertion}, which \textbf{marks} any unmarked expression into a marked
    expression, inserting error markings where appropriate.
\end{itemize}

In the interest of space, this overview focuses on the ``interesting'' cases; the whole system,
alongside the metatheorems discussed have been mechanized in the Agda proof assistant (discussed in
\cref{sec:calculus-agda}).

\subsection{Unmarked Expressions}
\label{sec:calculus-uexp}

\Cref{fig:calculus-syntax-uexp} gives the syntax of types, $\TMV$, and unmarked expressions, $\EMV$,
which extend the simply typed lambda calculus with $\TNum$ and $\TBool$ base types of numbers and
booleans. The number literal corresponding to the mathematical number $n$ is given by $\ENumMV$, and
there is a single addition operation on numeric expressions. $\ETrue$ and $\EFalse$ correspond to
the boolean values $\textsf{true}$ and $\textsf{false}$, and the branching expression
$\EIf{\EMV_1}{\EMV_2}{\EMV_3}$ gives an if-else form.

In addition, types are extended with an \emph{unknown type}, $\TUnknown$, which

\begin{figure}[htbp]
  \[\begin{array}{rrcl}
    \TMName  & \TMV  & \Coloneqq & \TUnknown \mid \TNum \mid \TBool \mid \TArrow{\TMV}{\TMV} \\
    \EMName  & \EMV  & \Coloneqq & x \mid \ELam{x}{\TMV}{\EMV} \mid \EAp{\EMV}{\EMV} \mid \ELet{x}{\EMV}{\EMV} \\
             &       & \mid         & \ENumMV \mid \EPlus{\EMV}{\EMV}
                       \mid           \ETrue \mid \EFalse \mid \EIf{\EMV}{\EMV}{\EMV}
  \end{array}\]
  \caption{Syntax of types and unmarked expressions.}
  \label{fig:calculus-syntax-uexp}
\end{figure}

Unmarked expressions are governed by a standard bidirectional gradual type system, as given in
\Cref{fig:calculus-typing-uexp}.

% TODO Not sure how explicitly we want to spell out the type system, in lieu of background
% section on bidirectional typing?

% \begin{figure}[htbp]
  \small\raggedright
  \judgbox{\ensuremath{\consistent{\TMV_1}{\TMV_2}}} $\TMV_1$ is consistent with $\TMV_2$
  %
  \begin{mathpar}
    \judgment{ }{
      \consistent{\TUnknown}{\TMV}
    }{TCUnknown1}

    \judgment{ }{
      \consistent{\TMV}{\TUnknown}
    }{TCUnknown2}

    \judgment{ }{
      \consistent{\TMV}{\TMV}
    }{TCRefl}

    \judgment{
      \consistent{\TMV_1}{\TMV_1'} \\
      \consistent{\TMV_2}{\TMV_2'} \\
    }{
      \consistent{\TArrow{\TMV_1}{\TMV_2}}{\TArrow{\TMV_1'}{\TMV_2'}}
    }{TCArr} \\
  \end{mathpar}

  \judgbox{\ensuremath{\matchedArrow{\TMV}{\TMV_1}{\TMV_2}}} $\TMV$ has matched arrow type $\TArrow{\TMV_1}{\TMV_2}$
  %
  \begin{mathpar}
    \judgment{ }{
      \matchedArrow{\TUnknown}{\TUnknown}{\TUnknown}
    }{TMAHole}

    \judgment{ }{
      \matchedArrow{\TArrow{\TMV}{\TMV}}{\TMV}{\TMV}
    }{TMAArr} \\
  \end{mathpar}

  \judgbox{\ensuremath{\TJoin{\TMV_1}{\TMV_2}}} is a \emph{partial} metafunction defined as follows:
  %
  \newcommand{\joinsTo}[3]{\ensuremath{\TJoin{#1}{#2} & = & #3}}
  \[\begin{array}{rcl}
    \joinsTo{\TUnknown}{\TMV}{\TMV} \\
    \joinsTo{\TMV}{\TUnknown}{\TMV} \\
    \joinsTo{\TNum}{\TNum}{\TNum} \\
    \joinsTo{\TBool}{\TBool}{\TBool} \\
    \joinsTo{(\TArrow{\TMV_1}{\TMV_2})}{(\TArrow{\TMV_1'}{\TMV_2'})}{\TArrow{(\TJoin{\TMV_1}{\TMV_1'})}{(\TJoin{\TMV_2}{\TMV_2'})}} \\
  \end{array}\]
  %
  \caption{Type consistency, matched arrow types, and type join.}
  \label{fig:calculus-type-judgments}
\end{figure}

% \begin{figure}[htbp]
  \small\raggedright
  \judgbox{\ctxSynType{\ctx}{\EMV}{\TMV}} $\EMV$ synthesizes type $\TMV$
  %
  \begin{mathpar}
    \judgment{
      \inCtx{\ctx}{x}{\TMV}
    }{
      \ctxSynType{\ctx}{x}{\TMV}
    }{USVar}

    \judgment{
      \ctxSynType{\extendCtx{\ctx}{x}{\TMV}}{\EMV}{\TMV_2}
    }{
      \ctxSynType{\ctx}{\ELam{x}{\TMV_1}{\EMV}}{\TArrow{\TMV_1}{\TMV_2}}
    }{USLam}

    \judgment{
      \ctxSynType{\ctx}{\EMV_1}{\TMV} \\\\
      \matchedArrow{\TMV}{\TMV_1}{\TMV_2} \\
      \ctxAnaType{\ctx}{\EMV_2}{\TMV_1}
    }{
      \ctxSynType{\ctx}{\EAp{\EMV_1}{\EMV_2}}{\TMV_2}
    }{USAp}

    % \judgment{
      % \ctxSynType{\ctx}{\EMV_1}{\TMV_1} \\
      % \ctxSynType{\extendCtx{\ctx}{x}{\TMV_1}}{\EMV_2}{\TMV_2}
    % }{
      % \ctxSynType{\ctx}{\ELet{x}{\EMV_1}{\EMV_2}}{\TMV_2}
    % }{USLet}

    \judgment{ }{
      \ctxSynType{\ctx}{\ENumMV}{\TNum}
    }{USNum}

    \judgment{
      \ctxAnaType{\ctx}{\EMV_1}{\TNum} \\
      \ctxAnaType{\ctx}{\EMV_2}{\TNum}
    }{
      \ctxSynType{\ctx}{\EPlus{\EMV_1}{\EMV_2}}{\TNum}
    }{USPlus}

    \judgment{ }{
      \ctxSynType{\ctx}{\ETrue}{\TBool}
    }{USTrue}

    \judgment{ }{
      \ctxSynType{\ctx}{\EFalse}{\TBool}
    }{USFalse}

    \judgment{
      \ctxAnaType{\ctx}{\EMV_1}{\TBool} \\
      \ctxSynType{\ctx}{\EMV_2}{\TMV_1} \\
      \ctxSynType{\ctx}{\EMV_3}{\TMV_2}
    }{
      \ctxSynType{\ctx}{\EIf{\EMV_1}{\EMV_2}{\EMV_3}}{\TMeet{\TMV_1}{\TMV_2}}
    }{USIf} \\
  \end{mathpar}
   
  \judgbox{\ctxAnaType{\ctx}{\EMV}{\TMV}} $\EMV$ analyzes against type $\TMV$
  %
  \begin{mathpar}
    \judgment{
      \matchedArrow{\TMV_3}{\TMV_1}{\TMV_2} \\
      \consistent{\TMV}{\TMV_1} \\\\
      \ctxAnaType{\extendCtx{\ctx}{x}{\TMV}}{\EMV}{\TMV_2}
    }{
      \ctxAnaType{\ctx}{\ECLam{x}{\TMV}{\EMV}}{\TMV_3}
    }{UALam}

    % \judgment{
      % \ctxSynType{\ctx}{\EMV_1}{\TMV_1} \\
      % \ctxAnaType{\extendCtx{\ctx}{x}{\TMV_1}}{\EMV_2}{\TMV_2}
    % }{
      % \ctxAnaType{\ctx}{\ELet{x}{\EMV_1}{\EMV_2}}{\TMV_2}
    % }{UALet}

    \judgment{
      \ctxAnaType{\ctx}{\EMV_1}{\TBool} \\
      \ctxAnaType{\ctx}{\EMV_1}{\TMV} \\
      \ctxAnaType{\ctx}{\EMV_2}{\TMV}
    }{
      \ctxAnaType{\ctx}{\ECIf{\EMV_1}{\EMV_2}{\EMV_3}}{\TMV}
    }{UAIf}

    \judgment{
      \ctxSynType{\ctx}{\EMV}{\TMV'} \\
      \consistent{\TMV}{\TMV'} \\\\
      \subsumable{\EMV}
    }{
      \ctxAnaType{\ctx}{\EMV}{\TMV}
    }{UASubsume}
  \end{mathpar}

  \caption{Unmarked expression type synthesis and analysis rules.}
  \label{fig:calculus-typing-uexp}
\end{figure}


\subsection{Marked Expressions}
\label{sec:calculus-mexp}

Marked expressions, $\ECMV$, extend the syntax of unmarked expressions with explicit \emph{error
markings}. As given in the syntax definition of \cref{fig:calculus-syntax-mexp}, there are three
such forms, which mirror the examples above:
%
\begin{itemize}
  \item $\ECUnbound{x}$, denoting the usage of an unbound variable.

  \item $\ECInconType{\ECMV}$, which occurs when the synthesized type of $\ECMV$ is inconsistent
    with the required type.

  \item $\ECInconBr{\ECMV_1}{\ECMV_2}{\ECMV_3}$, which occurs when the types of the two branches
    $\ECMV_2$ and $\ECMV_3$ are inconsistent.
\end{itemize}
%
Additional forms may arise as required by the language.

\begin{figure}[htbp]
  \[\begin{array}{rrcl}
    \ECMName & \ECMV & \Coloneqq & x \mid \ECLam{x}{\TMV}{\ECMV} \mid \ECAp{\ECMV}{\ECMV} \mid \ECLet{x}{\ECMV}{\ECMV} \\
             &       & \mid         & \ECNumMV \mid \ECPlus{\ECMV}{\ECMV}
                       \mid           \ECTrue \mid \ECFalse \mid \ECIf{\ECMV}{\ECMV}{\ECMV} \\
             &       & \mid         & \ECUnbound{x} \mid \ECInconType{\ECMV} \mid \ECInconBr{\ECMV}{\ECMV}{\ECMV}
  \end{array}\]
  \caption{Syntax of marked expressions}
  \label{fig:calculus-syntax-mexp}
\end{figure}

Marked expressions are also typed by a bidirectional system, which, for most forms, closely mirrors
the rules for corresponding unmarked forms. \Cref{fig:calculus-typing-mexp} contains a selection of
the ``interesting'' cases, particularly those concerned with error markings.

\begin{figure}[htbp]
  \small\raggedright
  \judgbox{\ctxSynType{\ctx}{\ECMV}{\TMV}} $\ECMV$ synthesizes type $\TMV$
  %
  \begin{mathpar}
    % \judgment{ }{
      % \ctxSynType{\ctx}{\ECEHole}{\TUnknown}
    % }{MSHole}

    % \judgment{
      % \inCtx{\ctx}{x}{\TMV}
    % }{
      % \ctxSynType{\ctx}{x}{\TMV}
    % }{MSVar}

    % \judgment{
      % \ctxSynType{\extendCtx{\ctx}{x}{\TMV}}{\ECMV}{\TMV_2}
    % }{
      % \ctxSynType{\ctx}{\ECLam{x}{\TMV_1}{\ECMV}}{\TArrow{\TMV_1}{\TMV_2}}
    % }{MSLam}

    \judgment{
      \ctxSynType{\ctx}{\ECMV_1}{\TMV} \\
      \matchedArrow{\TMV}{\TMV_1}{\TMV_2} \\\\
      \ctxAnaType{\ctx}{\ECMV_2}{\TMV_1}
    }{
      \ctxSynType{\ctx}{\ECAp{\ECMV_1}{\ECMV_2}}{\TMV_2}
    }{MSAp1}

    \judgment{
      \ctxSynType{\ctx}{\ECMV_1}{\TMV} \\
      \notMatchedArrow{\TMV} \\\\
      \ctxAnaType{\ctx}{\ECMV_2}{\TUnknown}
    }{
      \ctxSynType{\ctx}{\ECApNonMatched{\ECMV_1}{\ECMV_2}}{\TUnknown}
    }{MSAp2}

    % \judgment{
      % \ctxSynType{\ctx}{\ECMV_1}{\TMV_1} \\
      % \ctxSynType{\extendCtx{\ctx}{x}{\TMV_1}}{\ECMV_2}{\TMV_2}
    % }{
      % \ctxSynType{\ctx}{\ECLet{x}{\ECMV_1}{\ECMV_2}}{\TMV_2}
    % }{MSLet}

    % \judgment{ }{
      % \ctxSynType{\ctx}{\ECNumMV}{\TNum}
    % }{MSNum}

    % \judgment{
      % \ctxAnaType{\ctx}{\ECMV_1}{\TNum} \\
      % \ctxAnaType{\ctx}{\ECMV_2}{\TNum}
    % }{
      % \ctxSynType{\ctx}{\ECPlus{\ECMV_1}{\ECMV_2}}{\TNum}
    % }{MSPlus}

    % \judgment{ }{
      % \ctxSynType{\ctx}{\ECTrue}{\TBool}
    % }{MSTrue}

    % \judgment{ }{
      % \ctxSynType{\ctx}{\ECFalse}{\TBool}
    % }{MSFalse}

    % \judgment{
      % \ctxAnaType{\ctx}{\ECMV_1}{\TBool} \\
      % \ctxSynType{\ctx}{\ECMV_2}{\TMV_1} \\
      % \ctxSynType{\ctx}{\ECMV_3}{\TMV_2}
    % }{
      % \ctxSynType{\ctx}{\ECIf{\ECMV_1}{\ECMV_2}{\ECMV_3}}{\TJoin{\TMV_1}{\TMV_2}}
    % }{MSIf}

    \judgment{
      \notInCtx{\ctx}{x}
    }{
      \ctxSynType{\ctx}{\ECUnbound{x}}{\TUnknown}
    }{MSUnbound}

    \judgment{
      \ctxAnaType{\ctx}{\ECMV_1}{\TBool} \\
      \ctxSynType{\ctx}{\ECMV_2}{\TMV_1} \\\\
      \ctxSynType{\ctx}{\ECMV_3}{\TMV_2} \\
      \inconsistent{\TMV_1}{\TMV_2}
    }{
      \ctxSynType{\ctx}{\ECInconBr{\ECMV_1}{\ECMV_2}{\ECMV_3}}{\TUnknown}
    }{MSInconsistentBranches} \\
  \end{mathpar}

  \judgbox{\ctxAnaType{\ctx}{\ECMV}{\TMV}} $\ECMV$ analyzes against type $\TMV$
  %
  \begin{mathpar}
    \judgment{
      \matchedArrow{\TMV_3}{\TMV_1}{\TMV_2} \\
      \consistent{\TMV}{\TMV_1} \\\\
      \ctxAnaType{\extendCtx{\ctx}{x}{\TMV}}{\ECMV}{\TMV_2}
    }{
      \ctxAnaType{\ctx}{\ECLam{x}{\TMV}{\ECMV}}{\TMV_3}
    }{MALam1}

    \judgment{
      \notMatchedArrow{\TMV_3} \\
      \ctxAnaType{\extendCtx{\ctx}{x}{\TMV}}{\ECMV}{\TUnknown}
    }{
      \ctxAnaType{\ctx}{\ECInconType{\ECLam{x}{\TMV}{\ECMV}}}{\TMV_3}
    }{MALam2}

    \judgment{
      \matchedArrow{\TMV_3}{\TMV_1}{\TMV_2} \\
      \inconsistent{\TMV}{\TMV_1} \\\\
      \ctxAnaType{\extendCtx{\ctx}{x}{\TMV_1}}{\ECMV}{\TMV_2}
    }{
      \ctxAnaType{\ctx}{\ECInconType{\ECLam{x}{\TMV}{\ECMV}}}{\TMV_3}
    }{MALam3}

    % \judgment{
      % \ctxSynType{\ctx}{\ECMV_1}{\TMV_1} \\
      % \ctxAnaType{\extendCtx{\ctx}{x}{\TMV_1}}{\ECMV_2}{\TMV_2}
    % }{
      % \ctxAnaType{\ctx}{\ECLet{x}{\ECMV_1}{\ECMV_2}}{\TMV_2}
    % }{MALet}

    \judgment{
      \ctxAnaType{\ctx}{\ECMV_1}{\TBool} \\\\
      \ctxAnaType{\ctx}{\ECMV_1}{\TMV} \\
      \ctxAnaType{\ctx}{\ECMV_2}{\TMV}
    }{
      \ctxAnaType{\ctx}{\ECIf{\ECMV_1}{\ECMV_2}{\ECMV_3}}{\TMV}
    }{MAIf}

    \judgment{
      \ctxSynType{\ctx}{\ECMV}{\TMV'} \\
      \inconsistent{\TMV}{\TMV'} \\\\
      \subsumable{\ECMV}
    }{
      \ctxAnaType{\ctx}{\ECInconType{\ECMV}}{\TMV}
    }{MAInconsistentTypes}

    \judgment{
      \ctxSynType{\ctx}{\ECMV}{\TMV'} \\
      \consistent{\TMV}{\TMV'} \\\\
      \subsumable{\ECMV}
    }{
      \ctxAnaType{\ctx}{\ECMV}{\TMV}
    }{MASubsume}
  \end{mathpar}
  %
  \caption{Selected marked expression type synthesis and analysis rules.}
  \label{fig:calculus-typing-mexp}
\end{figure}


% TODO Not really sure how to explain the syntax and typing of marked expressions in an intuitive
% way in relation to unmarked expressions.

For unbound variables $\ECUnbound{x}$, nothing may be said about their types; hence, they synthesize
the unknown type. Inconsistent branch forms behave similarly, as inconsistent types have no join.

There are three rules concerned with the typing of lambda abstractions. MALam1, applies to
$\ECLam{x}{\TMV}{\ECMV}$ if the type $\TMV_3$ analysed against is a matched arrow type
$\TArrow{\TMV_1}{\TMV_2}$ and input type $\TMV_1$ is consistent with the binding ascription $\TMV$.
This corresponds to the lambda rule for unmarked expressions. However, we are additionally concerned
with the cases when $\TMV_3$ is not a matched arrow type or $\TMV_1$ is inconsistent with $\TMV$.
These are described by MALam2 and MALam3.

Alongside the standard subsumption rule, in which an expression that synthesizes a type may be
analysed against any consistent type, MAInconsistentTypes governs most occurrences of the
inconsistent types marking: $\ECInconType{\ECMV}$ analyzes against any type as long as $\ECMV$
synthesizes an inconsistent type. In both cases, we stipulate that subsumption may be applied to
$\ECMV$, written $\subsumable{\ECMV}$. In particular, lambda abstractions and if-else forms are
\emph{not} subsumable, formalizing these rules as ones of ``last resort'', preserving determinism of
typing.

Similarly to lambda abstractions, two rules govern application. In the unmarked syntax, the
application $\EAp{\EMV_1}{\EMV_2}$ was well-typed if the type $\TMV$ synthesized by $\EMV_1$ was a
matched arrow type. This corresponds to the MSAp1 rule for marked expressions. In the case that
$\TMV$ is not a matched arrow type, we require an inconsistent types marking around the

\subsection{Mark Insertion}

\subsection{Well-Formedness}
\label{sec:calculus-wellformedness}

\subsection{Alternative Formulations}

\subsection{Non-Gradual System}
\label{sec:calculus-non-gradual}

\subsection{Mechanization}
\label{sec:calculus-agda}

The metatheory of the calculus has been fully mechanized in the Agda proof assistant [citation]. The
mechanization's documentation contains more details regarding technical decisions made therein.
