\section{Related Work}
\label{sec:related}
\textbf{Type Inference and Unification} \emph{Gradual Typing with Unification Based Inference} explores a version of unification for gradually typed languages based on Huet's unification algorithm ~\cite{GradualInfer} ~\cite{Huet}. Other common approaches to unification are generally based on Hindley-Milner type inference ~\cite{MilnerInfer}. These algorithms are efficient, but can fail to solve constraints in the face of inconsistencies.

% Complex techniques, e.g. based on machine learning or manual weighting, have been proposed to heuristically localize these errors to expressions. 

\textbf{Correcting Sets and Assigning Blame} The generation of correcting sets is a common approach to error localization. The correcting set is assigned blame for failure in type inference \cite{sherrloc} \cite{typeinferDif} \cite{Pavlinovic2015}. Other approaches to assigning blame include statistically derived approaches that leverage machine learning to predict the best candidates for blame \cite{SeidelBlame}.

\textbf{Dynamic Witnesses} Explanations for type errors can be well illustrated by providing sample inputs (dynamic witnesses) that elicit runtime errors. With this approach, one can generate graphs for visualizing the execution of witnesses and heuristically identify the source of errors with around 70\% accuracy \cite{Seidel2016}.



