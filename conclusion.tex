\section{Conclusion}
\label{sec:conclusion}

\begin{quote}
    \emph{Nothing will ever be attempted if all possible objections must first be overcome.} 
    \begin{flushright}-- Samuel Johnson\end{flushright}
\end{quote}

Programming is increasingly a live collaboration between human programmers and sophisticated semantic services. These services need to be able to reason throughout the programming process, not just when the program is formally complete. This paper lays down rigorous type-theoretic foundations for doing just that. Bidirectional type checking helps make the localization decisions we make systematic and predictable, and type hole inference shows how local and constraint-based type inference might operate hand-in-hand, rather than as alternatives. 
We hope that language designers will use the techniques introduced in this paper to consider more rigorously, perhaps even formally, the problems of type error localization and error recovery when designing future languages. 

% \section*{Acknowledgements}
% (omitted for review)

% Thank you for reading this paper!
